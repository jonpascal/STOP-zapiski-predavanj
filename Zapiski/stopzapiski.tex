\documentclass[10pt, a4paper]{article}
\usepackage[slovene]{babel}
\usepackage[T1]{fontenc}
\usepackage[utf8]{inputenc}
\usepackage{lmodern}
\usepackage{amsmath}
\usepackage{amsthm}
\usepackage{amssymb}
\usepackage{parskip}
\usepackage{pgfplots}
\usepackage{comment}
\usepackage{graphicx}
\usepackage{booktabs}
\usepackage{array}
\usepackage{makecell}
\usepackage{float}
%\usepackage{mdframed}
%\usepackage{thmbox}

%%%%%%%%%%%%%%%%%%%%%%%%%%%%%%%%%%%%%%%%%%%%%%%%%%%%%%%%%%%%%%%%%%%%%%

\usepackage[top=105pt, bottom=75pt, left=75pt, right=75pt]{geometry}
\setlength{\headsep}{15pt}
\setlength{\footskip}{45pt}

\usepackage{xcolor}
\usepackage{lipsum}

\usepackage{ifthen}
\usepackage{tikz}
\usetikzlibrary{calc}



%%%%%%%%%%%%%%%%%%%%%%%%%%%%%%%%%%%%%%%%%%%%%%%%%%%%%%%%%%%%%
\usepackage{tcolorbox}
\tcbuselibrary{skins, breakable}


%%%%%%%%%%%%%%%%%%%%%%%%%%%%%%
%%% with separate title
\xdefinecolor{thmTopColor}{RGB}{102, 102, 238}
\xdefinecolor{thmBackColor}{RGB}{245, 245, 255}

%%%%%%%%%%%%%%%%%%%%%%%%%%%%%%%%%%%%%%%%%%%%%%%%%%%%%%%%%%%%%%%%%%%%%%%%



\newtheorem{izr}{Izrek}[section]

\newenvironment{thmbox}[1]{%
  \tcolorbox[%
  empty,
  parbox=false,
  noparskip,
  enhanced,
  breakable,
  sharp corners,
  boxrule=-1pt,
  left=2ex,
  right=0ex,
  top=0ex,
  boxsep=1ex,
  before skip=2.5ex plus 2pt,
  after skip=2.5ex plus 2pt,
  colback=thmBackColor,
  colframe=white,
  coltitle=black,
  colbacktitle=thmBackColor,
  fonttitle=\bfseries,
  title=#1,
  titlerule=1pt,
  titlerule style=thmTopColor,
  overlay unbroken and last={%
    \draw[color=thmTopColor, line width=1.25pt]
    ($(frame.north west)+(.5em, -4.1ex)$)
    -- ($(frame.south west)+(.5em, 1ex)$) -- ++(2em, 0);
  }]
}{\endtcolorbox}

\newenvironment{izrek}[1][]{% before
  \refstepcounter{izr}%
  \ifthenelse{\equal{#1}{}}{%
    \begin{thmbox}{Izrek \theizr.}\itshape\hspace{-.75ex}%
  }{%
    \begin{thmbox}{Izrek \theizr%
        \hspace{.75ex}(\textnormal{#1}).}\itshape\hspace{-.75ex}
    }}
  {\end{thmbox}
}

{\theoremstyle{plain}
\newtheorem{posledica}[izr]{Posledica}
\newtheorem{trditev}[izr]{Trditev}

}

{\theoremstyle{definition}
\newtheorem{defi}{Definicija}[section]
\newtheorem{aksiom}{Aksiom}[section]
}

\newenvironment{noticeB}{%
  \tcolorbox[%
  notitle,
  empty,
  enhanced,  % delete the edge of the bottom page for a broken box
  breakable,
  coltext=black,
  colback=white, 
  fontupper=\rmfamily,
  parbox=false,
  noparskip,
  sharp corners,
  boxrule=-1pt,  % width of the box' edges
  frame hidden,
  left=7pt,  % inner space from text to the left edge
  right=7pt,
  top=5pt,
  bottom=5pt,
  % boxsep=0pt,
  before skip=2.5ex plus 2pt,
  after skip=2.5ex plus 2pt,
  borderline west = {1.5pt}{-0.1pt}{blue!30!black}, % second argument = offset
  overlay unbroken and last={%
    \draw[color=black, line width=1.25pt]
    ($(frame.south west)+(1.pt, -0.1pt)$) -- ++(2em, 0);
  }
  ]}
{\endtcolorbox}

\newenvironment{definicija}{\begin{noticeB}\begin{defi}}{%
    \end{defi}\end{noticeB}}

{\theoremstyle{remark}
\newtheorem*{opomba}{Opomba}
}

\newtheorem{zgled}{Zgled}[section]
\tcolorboxenvironment{zgled}{%
  enhanced jigsaw,
  boxrule=-1pt,
  colframe=gray!15,
  %borderline west={2pt}{0pt}{black},  % second argument is the offset
  interior hidden,
  sharp corners,
  breakable,
  before skip=2.5ex plus 2pt,
  after skip=2.5ex plus 2pt
}

%%%%%%%%%%%%%%%%%%%%%%%%%%%%%%%%%%%%%%%%%%%%%%%%%%%%%%%%%%%%%%%%%%%%%%%%
\newtheorem{lema}[izr]{Lema}
\tcolorboxenvironment{lema}{%
  enhanced jigsaw,
  boxrule=-1pt,
  sharp corners,
  colframe=white,
  borderline west={2pt}{0pt}{orange},  % second argument is the offset
  interior hidden,
  breakable,
  before skip=2.5ex plus 2pt,
  after skip=2.5ex plus 2pt
}

%%%%%%%%%%%%%%%%%%%%%%%%%%%%%%%%%%%%%%%%%%%%%%%%%%%%%%%%%%%%%%%%%%
\newenvironment{noticeC}{%
  \tcolorbox[%
  notitle,
  empty,
  enhanced,  % delete the edge of the bottom page for a broken box
  breakable,
  coltext=black, 
  fontupper=\rmfamily,
  parbox=false,
  noparskip,
  sharp corners,
  boxrule=-1pt,  % width of the box' edges
  frame hidden,
  left=7pt,  % inner space from text to the left edge
  right=7pt,
  top=5pt,
  bottom=5pt,
  % boxsep=0pt,
  before skip=2.5ex plus 2pt,
  after skip=2.5ex plus 2pt,
  %borderline west = {1.5pt}{-0.1pt}{gray}, % second argument = offset
  overlay unbroken and last={%
    %\draw[color=gray, line width=1.25pt]
    %($(frame.west)$);
    %\draw[color=gray, line width=1.25pt]
    %($(frame.east)$);
  },
  ]}
{\endtcolorbox}

\newenvironment{dokaz}%
  {\begin{noticeC}\begin{proof}}%
  {\end{proof}\end{noticeC}}

%%%%%%%%%%%%%%%%%%%%%%%%%%%%%%%%%%%%%%%%%%%%%%%%%%%%%%%%%%%%%%%%%%%%

\makeatletter
\newlength\xvec@height%
\newlength\xvec@depth%
\newlength\xvec@width%
\newcommand{\xvec}[2][]{%
  \ifmmode%
    \settoheight{\xvec@height}{$#2$}%
    \settodepth{\xvec@depth}{$#2$}%
    \settowidth{\xvec@width}{$#2$}%
  \else%
    \settoheight{\xvec@height}{#2}%
    \settodepth{\xvec@depth}{#2}%
    \settowidth{\xvec@width}{#2}%
  \fi%
  \def\xvec@arg{#1}%
  \def\xvec@dd{:}%
  \def\xvec@d{.}%
  \raisebox{.2ex}{\raisebox{\xvec@height}{\rlap{%
    \kern.05em%  (Because left edge of drawing is at .05em)
    \begin{tikzpicture}[scale=1]
    \pgfsetroundcap
    \draw (.05em,0)--(\xvec@width-.05em,0);
    \draw (\xvec@width-.05em,0)--(\xvec@width-.15em, .075em);
    \draw (\xvec@width-.05em,0)--(\xvec@width-.15em,-.075em);
    \ifx\xvec@arg\xvec@d%
      \fill(\xvec@width*.45,.5ex) circle (.5pt);%
    \else\ifx\xvec@arg\xvec@dd%
      \fill(\xvec@width*.30,.5ex) circle (.5pt);%
      \fill(\xvec@width*.65,.5ex) circle (.5pt);%
    \fi\fi%
    \end{tikzpicture}%
  }}}%
  #2%
}
\makeatother

% --- Override \vec with an invocation of \xvec.
\let\stdvec\vec
\renewcommand{\vec}[1]{\xvec[]{#1}}
% --- Define \dvec and \ddvec for dotted and double-dotted vectors.
\newcommand{\dvec}[1]{\xvec[.]{#1}}
\newcommand{\ddvec}[1]{\xvec[:]{#1}}
\newcommand{\stcomp}[1]{{#1}^{\mathsf{c}}}

%%%%%%%%%%%%%%%%%%%%%%%%%%%%%%%%%%%%%%%%%%%%%%%%%%%%%%%%%%%%%%%%%%%%

\newcommand{\N}{\mathbb {N}}
\newcommand{\Z}{\mathbb {Z}}
\newcommand{\Q}{\mathbb {Q}}
\newcommand{\R}{\mathbb {R}}
\newcommand{\C}{\mathbb {C}}
\newcommand{\F}{\mathbb {F}}
\newcommand{\zap}[1]{(#1_n)_{n=1} ^{\infty}}
\newcommand{\podzap}[1]{(#1_{n_j})_{n=1 ^{\infty}}}
\newcommand{\limzap}[1]{\lim_{n \to \infty} {#1}}
\newcommand{\ve}[1]{\overrightarrow{#1}}
\newcommand{\vectors}[2]{\vec{{#1}_1},\vec{{#1}_2}, \dots \vec{{#1}_{#2}}}
\newcommand{\scalars}[2]{{#1}_1, {#1}_2, \dots, {#1}_{#2}}
\newcommand{\im}{\textup{\text{im}}\,}
\newcommand{\rang}{\textup{\text{rang}}}
\newcommand{\isom}{\stackrel{\sim}{=}}
\newcommand{\quot}[2]{{\raisebox{.2em}{$#1$}\left/\raisebox{-.2em}{$#2$}\right.}}
\newcommand{\sprod}[2]{\langle {#1},{#2} \rangle}
\newcommand{\topo}[1]{\mathcal{#1}}
\newcommand{\cl}{\mathrm{Cl}}
\newcommand{\inte}{\mathrm{Int}}
\newcommand{\fr}{\mathrm{Fr}}
\DeclareMathOperator{\di}{d\!}

\newcolumntype{C}[1]{>{\centering\let\newline\\\arraybackslash\hspace\hspace{0pt}}m{#1}}

\setlength{\parskip}{1em}

\begin{document}

\title{SPLOŠNA TOPOLOGIJA - ZAPISKI}
\author{Gal Anton Gorše}
\date{}
\maketitle

\section*{Uvod}

Topologija je v osnovi študij zveznih preslikav in s tem povezanih pojmov.
Intuitivno se zdi, da je funkcija $F: X \rightarrow Y$ zvezna, če točke, ki so si blizu v $X$ preslika v točke,
ki so si blizu v $Y$. Pojem bližine je seveda relativen, zato vedno formuliramo pogoje, ki jih je potrebno izpolniti
za vsako možno natančnost. Oglejmo si primer definicije limite zaporedja in zveznosti funkcije $f : \R \rightarrow \R$, ki smo ga spoznali že lani.

\begin{definicija}
  Zaporedje $\zap{x}$ ima limito $l = \lim_{n \to \infty} x_n$ natanko tedaj, ko za $\forall \varepsilon > 0$ 
  obstaja $N \in \N$, da velja $n > N \Rightarrow |x_n - l| < \varepsilon$. 
\end{definicija}

\begin{definicija}
  Funkcija $f : \R \to \R$ je zvezna v $a \in \R$, če za $\forall \varepsilon$ obstaja tak $\delta > 0$,
  da velja $|x - a| < \delta \Rightarrow |f(x) - f(a)|< \varepsilon$.
\end{definicija}

Absolutna vrednost razlike, ki se pojavlja v teh dveh definicijah, pa je le zelo poseben primer razdalje med točkami,
zato zveznost naravno formuliramo v metričnih prostorih.

\begin{definicija}
  Metrika na množici $X$ je taka preslikava $d : X \times X \to [0, + \infty)$, da velja:
  \begin{itemize}
    \item $d(x, x') = 0 \Leftrightarrow x = x'$,
    \item $d(x, x') = d(x', x)$ in 
    \item $d(x, x') + d(x', x'') \geq d(x, x'')$.
  \end{itemize}
\end{definicija}

\begin{definicija}
  Preslikava $f: (X, d) \to (X', d')$ je zvezna v točki $a \in X$, če za $\forall \varepsilon > 0$ 
  obstaja $\delta > 0$, da velja $d(x, a) < \delta \Rightarrow d'(f(x), f(a)) < \varepsilon$.
  Preslikava $f$ je zvezna, če je zvezna v vseh točkah.
\end{definicija}

Čeprav nam metrika omogoča, da preprosto obravnavamo bližino in zveznost, smo v geometriji in analizi 
že spoznali tudi primere, ki se jih ne da povsem naravno opisati v jeziku metrike.
Enostaven primer je na primer, če vzamemo interval $[0, 1]$ (ponazorjen z daljico)
in njegovi krajišči skupaj zalepimo, da dobimo krožnico.
Oba prostora sta metrična, vendar se metrike na krožnici ne da definirati tako, da bi izhajala iz metrike na intervalu.
Nasprotno pa lahko odprte množice na krožnici dobimo z "`lepljenjem"' odprtih množic na intervalu.

Nekateri osrednji pojmi iz analize niso metrične narave. Primer je konvergenca Taylorjeve vrste k limitni funkciji.
Delne vsote vrste $\sin x = x - \frac{x^3}{6} + \frac{x^5}{120} - \dots$ so polinomi.
Ker je funkcija sinus omejena na $\R$, polinomi pa niso omejeni, konvergence delnih vsot vrste proti limitni funkciji ni mogoče predstaviti
kot konvergenco na neki metriki (Taylorjevi polinomi konvergirajo enakomerno le na vsakem zaprtem intervalu konvergenčnega območja vrste).
Lahko pa jo opišemo kot konvergenco glede na neko topologijo, ki ji pravimo kompaktno-odprta topologija 
in ki se izkaže za naravno topologijo na prostoru zveznih funkcij.

\section{Prostori in preslikave}

\subsection{Topološki prostori}

\begin{definicija}
  Naj bo $X$ množica. Topologija na $X$ je družina $\mathcal{T}$ podmnožic $X$,
  ki zadošča pogojem:
  \begin{enumerate}
    \item $\emptyset, X \in \mathcal{T}$,
    \item poljubna unija elementov $\mathcal{T}$ je tudi element $\mathcal{T}$ in 
    \item končen presek elementov $\mathcal{T}$ je tudi element $\mathcal{T}$.
  \end{enumerate}
  Topološki prostor je par $(X, \mathcal{T})$. Elemente $\mathcal{T}$ imenujemo odprte množice. 
\end{definicija}

\begin{zgled}
    Metrika $d$ na množici $X$ inducira topologijo $\mathcal{T}_d$ na $X$.
    Tedaj velja $U \in \mathcal{T}_d$ natanko tedaj, ko za $\forall x \in U$
    obstaja $r > 0$, da velja $K(x, r) \subseteq U$.
    Pravimo, da je $\mathcal{T}_d$ topologija, porojena z metriko $d$.
    Prostor $(X, \mathcal{T})$ je metrizabilen, če je $\mathcal{T} = \mathcal{T}_d$ za neko metriko $d$.
    Omenimo še, da lahko na metričnem prostoru $(X, d)$ metriko $d$ nadomestimo z omejeno metriko
    $\overline{d} (x, x') = \min \{d(x, x'), 1\}$. Pri tem se topologija ne spremeni: $\mathcal{T}_d = \mathcal{T}_{\overline{d}}$.
\end{zgled}

\begin{zgled}
  Vzemimo množico $\R$ in definirajmo topologijo $\mathcal{T} = \{\emptyset, \R, (x, \infty), x\in \R\}$
    (enostavno je preveriti, da ta množica res predstavlja topologijo).
    Poraja se vprašanje, ali je topološki prostor $(\R, \mathcal{T})$ metrizabilen.
    Če sumimo, da bo odgovor negativen, moramo najti neko invariantno lastnost metričnih prostorov, ki ne velja za ta topološki prostor.
    V tem primeru bo to obstoj disjunktnih okolic v metričnih prostorih: če je $x \neq x'$ in $d(x, x') = r$,
    potem sta okolici $K\left(x, \frac{r}{2}\right)$ in $K\left(x', \frac{r}{2}\right)$ disjunktni, v našem topološkem prostoru pa ni disjunktnih odprtih množic.
\end{zgled}

\begin{zgled}
    $\mathcal{T} = \{\emptyset, X\}$ je topologija na $X$ in ji pravimo kar trivialna topologija.
    Po drugi strani pa je tudi $\mathcal{T} = P(X)$ topologija na $X$ in ji pravimo diskretna topologija.
    Trivialna topologija ni metrizabilna (če ima $X$ več kot eno točko), diskretna pa je zmeraj.
    Njena metrika je tako imenovana diskretna metrika $$d(x, x') = \begin{cases}
      0 ;& x = x'\\
      1 ;& x \neq x'
    \end{cases}.$$
\end{zgled}

\begin{opomba}
  Omenimo še dve opombi glede definicije topologije.
  \begin{itemize}
    \item Unija prazne družine je $\emptyset$, njen presek pa $X$,
    zato je formalno gledano v definiciji topologiji dovolj zahtevati le točki $(2)$ in $(3)$.
    Topologija na $X$ je torej družina podmnožic $X$, ki je zaprta za poljubne unije in končne preseke.
    \item Zahtevo $(3)$ lahko nadomestimo s $(3')$: presek dveh elementov $\mathcal{T}$ je tudi element $\mathcal{T}$.
    Vendar pa moramo potem izrecno zahtevati $X \in \mathcal{T}$.
  \end{itemize}
\end{opomba}

Pojave iz metričnih prostorov lahko prevedemo v topološki jezik.

\begin{table}[bth!]
  \centering
  \begin{tabular}{ccc}
    \toprule
                                  & metrični prostor $(X, d)$                                              & topološki prostor $(X, \mathcal{T})$\\
    \midrule
    okolica                       & \makecell{krogla s središčem v $x$\\ ali množica, ki jo vsebuje}       & \makecell{$A \subseteq X$ je okolica $x$, če obstaja\\ $U \in \mathcal{T}$, da je $x \in U \subseteq A$.\\$U$ je okolica vsake svoje meje} \\         
    \midrule
    $x$ je notranja točka $A$    & $A$ vsebuje neko okolico $x$                                           & enako\\
    \midrule
    $x$ je mejna točka $A$        & vsaka okolica $x$ seka $A$ in $X \setminus A$                          & enako\\
    \midrule
    notranjost $A$ ($\inte\, A$, $\overset{\circ}{A}$)    & vse notranje točke $A$                                                 & \makecell{največja odprta podmnožica\\ $A$ ali unija vseh elementov $\mathcal{T}$,\\vsebovanih v $A$}\\
    \midrule
    meja $A$ ($\fr \, A$)                     & vse mejne točke                                                        & $\overline{A} \setminus \inte\, A$\\
    \midrule
    zaprtje $A$ ($\cl\, A$, $\overline{A}$)  & notranjost $A$ in meja $A$                                             & \makecell{najmanjša zaprta množica,\\ ki vsebuje $A$ ali presek vseh\\zaprtih množic, ki vsebujejo $A$}\\
    \midrule 
    $A$ je odprta                 & vse točke $A$ so notranje                                              & $A \in \mathcal{T}$\\
    \midrule
    $A$ je zaprta                 & $A$ vsebuje vse mejne točke                                            & $\stcomp{A} \in \mathcal{T}$\\
    \midrule
    $X$ je omejena                & premer $X$ je končen                                                   & omejenost ni topološki pojem\\
    \midrule
    $X$ je poln                   & \makecell{vsako Cauchyjevo za-\\poredje ima limito}                    & ni topološki pojem\\
    \midrule
    $x$ je stekališče $A$         & \makecell{vsaka okolica $x$ seka $A \setminus \{x\}$}                  & enako\\
    \midrule
    $x = \limzap{x_n}$            & \makecell{vsaka okolica $x$ vsebuje\\skoraj vse točke $\zap{x}$}                  & enako\\
    \midrule
    $X$ je kompakten              & vsako zaporedje ima stekališče     & \makecell{vsako odprto pokritje prostora\\ ima končno podpokritje} \\
    \bottomrule
  \end{tabular}
  \caption{Pojmi iz metričnih prostorov in njihovi "`prevodi"' v jezik topologije}
\end{table}

\begin{opomba}
  Ista množica je lahko hkrati odprta v enem prostoru in zaprta v drugem: primer je 
  recimo list papirja v ravnini $\R^2$ in v prostoru $\R^3$.
  Zato je odprtje in zaprtje množice $A$ bolje označiti kot $\inte_X \; A$ in $\cl_X \; A$, razen ko je prostor $X$ očiten.
\end{opomba}

V topoloških prostorih je poudarjena preprosta zveza med odprtimi in zaprtimi množicami: ene so komplementi drugih.
Posledično lahko topologijo vpeljemo tako, da predpišemo, katere množice so zaprte.
Izkaže se, da je to marsikdaj bližje intuiciji.
Naj bo $\mathcal{Z}$ neka družina podmnožic $X$. Denimo, da za $\mathcal{Z}$ veljajo točke:
\begin{enumerate}
  \item $\emptyset, X \in \mathcal{Z}$,
  \item poljuben presek elementov $\mathcal{Z}$ je v $\mathcal{Z}$ in 
  \item končna unija elementov $\mathcal{Z}$ je v $\mathcal{Z}$.
\end{enumerate}
Potem je $\mathcal{Z}$ družina vseh zaprtih množic $X$ in $\mathcal{T} = \{U \subseteq X\ |\ \stcomp{U} \in \mathcal{Z}\}$
je topologija na $X$. 
Tedaj pravimo, da smo topologijo podali z zaprtimi množicami.

\begin{zgled}
  Kaj dobimo, če zahtevamo, da so točke zaprte? Potem so vse končne množice in $X$ zaprte.
  Tu pa je dovolj, da vidimo, da lahko množico, ki vsebuje $X$ in vse njegove končne podmnožice,
  razglasimo za $\mathcal{Z}$. Tako dobimo topologijo končnih komplementov
  $\mathcal{T}_{KK}$, ki vsebuje $X$ in vse njegove podmnožice, ki imajo končen komplement.
  Če je $X$ končna, ima diskretno topologijo, sicer pa ni metrizabilna.
\end{zgled}

\subsection{Zvezne preslikave}

Zveznost funkcije $f: X \to Y$ lahko opišemo takole: za vsak $y = f(x) \in Y$
in vsako okolico $V$ točke $y$ obstaja okolica točke $x$, ki jo $f$ preslika v $V$.
Drugače povedano, vsaka točka, ki se preslika v notranjo točko nekega $V \subseteq Y$,
je notranja točka $f^{-1} (V)$. To pa pomeni, da je praslika vsake odprte podmnožice $Y$ odprta podmnožica $X$.
To je (re)definicija zveznosti na način, ki ne omenja posameznih točk, temveč le odprte množice.

\begin{definicija}
  Funkcija $f: (X, \mathcal{T}) \to (X', \mathcal{T}')$ je zvezna, če je praslika 
  vsake odprte podmnožice $X'$ odprta, oziroma če velja $V \in \mathcal{T}' \Rightarrow f^{-1} (V) \in \mathcal{T}$.
\end{definicija}

\begin{zgled}
  Oglejmo si nekaj primerov zveznih funkcij med topološkimi prostori.
  \begin{itemize}
    \item Zvezne funkcije med metričnimi prostori so zvezne tudi glede na z metrikami porojene topologije.
    To seveda sledi iz načina, na katerega smo definirali zveznost na topoloških prostorih.
    \item Trivialna topologija: $f: (X, \mathcal{T}) \to (Y, \mathcal{T}_{\text{trivialna}})$ je vedno zvezna.
    \item Diskretna topologija: $f: (X, \mathcal{T}_{\text{diskretna}}) \to (Y, \mathcal{T})$ je vedno zvezna.
    \item Konstantna funkcija: za $y \in Y$ je $\mathrm{const}_y : (X, \mathcal{T}) \to (Y, \mathcal{T}')$ vedno (včasih edina) zvezna.
    \item Identiteta: $\mathrm{Id}: (X, \mathcal{T}) \to (X, \mathcal{T'})$ je zvezna $\Leftrightarrow$ velja $\mathcal{T}' \subseteq \mathcal{T}$.
  \end{itemize}
\end{zgled}

\begin{trditev}
  Kompozitum zveznih preslikav je zvezna preslikava.
\end{trditev}

Množico vseh zveznih preslikav med $(X, \mathcal{T})$ in $(Y, \mathcal{T}')$ označimo s $C((X, \mathcal{T}), (Y, \mathcal{T}'))$ ali krajše $C(X, Y)$.
Če je $Y = \R$ z običajno (evklidsko) topologijo, potem je $C(X) = C(X, \R)$.
Dogovorimo se, da topološkemu prostoru rečemo kar  `prostor', zvezni preslikavi pa `preslikava'.

\begin{izrek}[Karakterizacija zveznosti]
  Naslednje trditve so ekvivalentne:
  \begin{enumerate}
    \item $f: (X, \mathcal{T}) \to (Y, \mathcal{T}')$ je zvezna.
    \item Za vsak odprt $V \subseteq Y$ je $f^{-1} (V)$ odprt v $X$.
    \item Za vsak zaprt $B \subseteq Y$ je $f^{-1} (B)$ zaprt v $X$.
    \item Za vsak $A \subseteq X$ velja $f(\overline{A}) \subseteq \overline{(f(A))}$.
  \end{enumerate}
\end{izrek}

\begin{dokaz}
  Ekvivalenca $(1) \Leftrightarrow (2)$ je definicija zveznosti, $(2) \Leftrightarrow (3)$ pa 
  sledi direktno iz dejstva $f^{-1} (\stcomp{B}) = \stcomp{f^{-1}(B)}$. Dokazati moramo le še zadnjo ekvivalenco.

  $(3) \Rightarrow (4)$ Naj bo $A \subseteq X$. Potem je $A \subseteq f^{-1} (f(A)) \subseteq f^{-1} \left(\overline{f(A)}\right)$
  in slednja množica je zaprta po točki $(3)$. Od tod sledi $\overline{A} \subseteq f^{-1} \left(\overline{f(A)}\right)$ in posledično $f(\overline{A}) \subseteq \overline{f(A)}$.

  $(4) \Rightarrow (3)$ Naj bo $B \subseteq Y$ zaprta. Potem je $f \left(\overline{f^{-1} (B)} \right) \subseteq \overline{f \left(f^{-1} (B)\right)} \subseteq \overline{B} = B$
  in posledično $\overline{f^{-1} (B)} \subseteq f^{-1} (B)$. To pa je možno le, ko sta ti množici enaki, torej je $f^{-1} (B)$ zaprt.
\end{dokaz}

V metričnih prostorih je $f$ zvezna natanko tedaj, ko velja $\lim_{i \to \infty} f(x_i) = f\left( \lim_{i \to \infty} x_i\right)$.
Primerjajmo to s formulacijo $f(\overline{A}) \subseteq \overline{f(A)}$.
Ali lahko zveznost preslikav karakteriziramo z limitami tudi v topoloških prostorih?
Odgovor je ne, saj če vzamemo množico $X$ s trivialno topologijo, so vse točke $X$ limite kateregakoli zaporedja.
Lahko pa se zgodi tudi nasprotno in imamo zelo bogato topologijo, v kateri se morda neki točki približujemo na neštevno mnogo načinov.
V takih robnih primerih se zgodi, da množici 
$\{\text{limite vseh zaporedij v $A$}\} \subseteq \overline{A}$
nista enaki.

\subsection{Homeomorfizem}

Kdaj bi rekli, da sta topološka prostora $(X, \mathcal{T})$ in $(X', \mathcal{T}')$ ekvivalentna?
Intuitivno potrebujemo funkcijo $f: X \to X'$, ki je bijekcija in ki obenem določa bijekcijo med 
$\mathcal{T}$ in $\mathcal{T}'$.

\begin{definicija}
  Funkcija $f: (X, \mathcal{T}) \to (X', \mathcal{T}')$ je homeomorfizem, če je $f: X \to X'$ bijekcija
  in če $f$ inducira bijekcijo med $\mathcal{T}$ in $\mathcal{T}'$. Tedaj pišemo $(X, \mathcal{T}) \approx (X', \mathcal{T}')$
  in pravimo, da sta prostora homeomorfna.
\end{definicija}

Če bijektivna funkcija $f: X \to X'$ inducira bijekcijo med $\mathcal{T}$ in $\mathcal{T}'$, mora veljati dvoje:
za vsako odprto množico $U \in \mathcal{T}$ mora veljati $f(U) \in \topo{T}'$ in obratno, 
za vsako odprto množico $V \in \topo{T}'$ mora biti $f^{-1} (V) \in \topo{T}$. 

\begin{trditev}
  Naslednje izjave so ekvivalentne.
  \begin{enumerate}
    \item $f: X \to X'$ je homeomorfizem.
    \item $f$ je bijekcija in $f$ ter $f^{-1}$ sta zvezni.
    \item $f$ je zvezna, odprta bijekcija.
    \item $f$ je zvezna, zaprta bijekcija.
  \end{enumerate}
\end{trditev}

\begin{opomba}
  Funkcija $f$ je odprta, če velja $U \in \mathcal{T} \Rightarrow f(U) \in \mathcal{T}'$, in 
  zaprta, če velja $\stcomp{A} \in \mathcal{T} \Rightarrow \stcomp{f(A)} \in \mathcal{T}'$.
\end{opomba}

Če je $f$ bijektivna, potem je $f$ odprta natanko tedaj, ko je zaprta.
Če pa $f$ ni bijekcija, sta pojma različna. Protiprimer je recimo konstantna funkcija (ta je vedno zaprta, odprta pa le tedaj,
ko slika v izolirano točko).

\begin{zgled}
  V realni množici velja $[a, b] \approx [c, d]$ s homeomorfizmom $f(x) = \frac{d - c}{b - a}(x-a) + c$.
  Ista formula velja za homeomorfizem med odprtimi in polodprtimi intervali. Za homeomorfizem med $(-1, 1)$ in $\R$
  obstaja več možnosti, kot na primer $f(x) = \tan \left(\frac{\pi}{2} x\right)$ ali še bolje $f(x) = \frac{x}{1 - |x|}$.
\end{zgled}

Vsak interval na realni osi (vključno z neskončnimi) je homeomorfen enemu izmed intervalov 
$[-1, 1]$, $[-1, 1)$ ali $(-1, 1)$. Intervala $[-1, 1]$ in $(-1, 1)$ nista homeomorfna, saj bi zvezna preslikava 
kompaktno množico (v $\R^n$ zaprt interval) preslikala v nekompaktno množico (nezaprt interval). Enako seveda velja tudi 
za $[-1, 1]$ in $[-1, 1)$. Enostavno pa je preveriti tudi, da ne obstaja homeomorfizem $f$ med $[-1, 1)$ in $(-1, 1)$,
če opazujemo točko $f(1) = a \in (-1, 1)$.

Topološka lastnost je lastnost $(X, \mathcal{T})$, ki se ohranja pri homeomorfizmih, na primer kompaktnost in povezanost
(primer lastnosti, ki pa ni topološka, je recimo polnost).

Sedaj definirajmo naslednje pojme:
  \begin{itemize}
    \item zaprta enotska krogla $B^n = \{\vec{x} \in \R^n\ |\ |\vec{x}| \leq 1\}$ v prostoru $\R^n$,
    \item odprta enotska krogla $\overset{\circ}{B^n} = \{\vec{x} \in \R^n\ |\ |\vec{x}| \leq 1\}$ in 
    \item enotska sfera $S^{n-1} = \{\vec{x} \in \R^n\ |\ |\vec{x}| = 1\}$.
  \end{itemize}
Potem je $\overset{\circ}{B^n}$ homeomorfna $\R^n$, saj med njima obstaja homeomorfizem $f(\vec{x}) = \frac{\vec{x}}{1 - |\vec{x}|}$
z inverzom $f(\vec{x}) = \frac{\vec{x}}{1 + |\vec{x}|}$. S kratkim premislekom pa lahko dokažemo tudi naslednjo trditev.

\begin{trditev}
  Naj bo $A \subseteq \R^n$ zaprta, omejena in konveksna z neprazno notranjostjo.
  Potem velja $A \approx B^n$.
\end{trditev}

Zdi se naravno, da če sferi $S^{n - 1}$ odstranimo eno samo točko, lahko njen "`plašč"' razprostremo na celoten prostor $\R^{n - 1}$.
Dokažimo na primer, da je množica $S^{n - 1} \setminus \{(0, \dots, 0, 1)\}$ homeomorfna prostoru $\R^{n-1}$.

\begin{zgled}
  Oglejmo si primer $n = 2$. Tukaj lahko točke na enotski krožnici v ravnini $\R^2$ projiciramo
  na abscisno os glede na točko $(0, 1)$.
  Dobimo formulo $f(x, y) = \frac{x}{1 - y}$, ki ima inverz $f^{-1} (u) = \left(\frac{2u}{u^2 + 1}, \frac{u^2 - 1}{u^2 + 1}\right)$.
  Pri podobnih primerih si pogosto pomagamo z raznimi projekcijami.
\end{zgled}

Sedaj se vrnimo v splošen primer. Množico $S^{n - 1} \setminus \{(0, \dots, 0, 1)\}$ preslikamo v $\R^{n - 1}$
s homeomorfizmom $f(\vec{x}, y) = \frac{\vec{x}}{1 - y}$, ki ima inverz $f^{-1} (\vec{x}) = \left(\frac{2 \vec{x}}{|\vec{x}|^2 + 1}, \frac{|\vec{x}|^2 - 1}{|\vec{x}|^2 + 1}\right)$.
Preslikavi $f$ rečemo stereografska projekcija.
Pokazali smo, da je $S^{n - 1} \setminus \{(0, \dots, 0, 1)\} \approx \R^{n - 1}$.
Enak rezultat bi dobili, če bi iz sfere izrezali katerokoli točko.
Sklepamo, da ima vsaka točka $S^{n - 1}$ okolico, ki je homeomorfna $\R^{n - 1}$, zato pravimo,
da je $S^{n - 1}$ lokalno homeomorfna prostoru $\R^{n - 1}$.
Pogosto rečemo, da $S^n$ določimo tako,
da $\R^n$ dodamo točko v neskončnosti.
Prostorom, ki so lokalno homeomorfni kakemu evklidskemu prostoru, pravimo mnogoterost.

\begin{zgled}
  Oglejmo si primer, ki smo ga že obravnavali: interval preslikamo v krožnico.
  Naj bo $S^1 \subseteq \C$ in imamo preslikavo $f: [0, 2\pi) \to S^1$ s predpisom $f(t) = e^{it}$.
  Ta preslikava je očitno bijektivna in zvezna, ni pa zaprta, saj slika intervala $[\pi, 2\pi)$ (zaprta podmnožica $[\pi, 2\pi)$)
  ni zaprta v $S^1$.
\end{zgled}

V konkretnih primerih ni težko pokazati, ali je dana preslikava homeomorfizem.
Zaradi tega običajno ni težko za dva homeomorfna prostora pokazati, ali sta homeomorfna:
dovolj je, če poiščemo eksplicitni homeomorfizem.
Veliko težje je pokazati, da dana prostora nista homeomorfna. Pri tem imajo ključno vlogo topološke lastnosti.

\begin{definicija}
  Za neko lastnost prostora $X$ pravimo, da je topološka lastnost, če jo imajo tudi vsi prostori, ki so homeomorfni z $X$.
\end{definicija}

Primeri (preprostih) topoloških lastnosti, ki smo jih že srečali, so končnost ali števnost točk, diskretnost, kompaktnost ali povezanost prostora.
Primer lastnosti, ki pa ni topološka, je recimo omejenost.

\subsection{Baze in predbaze}

Topološke pojme smo doslej definirali z opredelitvijo družine vseh odprtih množic,
vendar pa bi si poenostavili premisleke, če bi zadoščalo preveriti zveznost ali odprtost preslikave na kakšni manjši in
bolj obvladljivi družini odprtih množic.
Idejo za rešitev dobimo pri metričnih prostorih:
funkcija $f$ je zvezna v točki $x$, če za vsako $\varepsilon$-okolico $V$ točke $f(x)$ praslika $f^{-1} (V)$
vsebuje kako $\delta$-okolico točke $x$.
Zadošča torej, če zveznost preizkusimo le na majhnih okolicah.

\begin{definicija}
  Lokalna baza okolic pri $x \in X$: $\mathcal{B}_x \subseteq \mathcal{T}$ so odprte okolice $x$
  in velja, da za vsak $U \in \mathcal{T}$, ki vsebuje $x$, obstaja $B \in \mathcal{B}_x$, da je $B \subseteq U$.
\end{definicija}

\begin{trditev}
  Naj bo $f: X \to X'$ poljubna funkcija, $\mathcal{B}_x$ neka baza okolic točke $x \in X$ in $\topo{B}_{f(x)}'$
  neka baza okolic točke $f(x) \in X'$.
  Funkcija $f$ je zvezna v $x$ natanko takrat, ko za vsak $B' \in \topo{B}_{f(x)}'$ praslika $f^{-1} (B')$
  vsebuje neki element $B \in \topo{B}_x$.
\end{trditev}
Na podoben način karakteriziramo zveznost na celem prostoru.

\begin{definicija}
  Za neko družino odprtih množic $\mathcal{B} \subseteq \mathcal{T}$ rečemo, da je baza topologije $\mathcal{T}$,
  če lahko vsak element $\mathcal{T}$ dobimo kot unijo elementov $\mathcal{B}$. 
\end{definicija}

\begin{zgled}
  Kanonični primer: v metričnem prostoru lahko za bazo topologije
  vzamemo odprte krogle.
  Lahko smo varčni in vzamemo le majhne krogle, ali samo krogle z racionalnimi polmeri,
  ali le krogle s polmerom $\frac{1}{n}$ za $n = 1, 2, 3, \dots$.
  Lahko pa v $\R^n$ vzamemo le krogle s središčem v $\Q^n$ (in celo samo z racionalnimi polmeri).
\end{zgled}

Zapis elementov $\mathcal{T}$ kot unije elementov $\mathcal{B}$ ni enoličen,
zato je baza topologije analogna ogrodju vektorskega prostora.

\begin{trditev}
  Naj bo topološki prostor $(X, \mathcal{T})$ in $\mathcal{B}$ baza $\mathcal{T}$.
  \begin{enumerate}
    \item $A \subseteq X$ je odprta natanko tedaj, ko za vsak $x \in A$ obstaja $B \in \mathcal{B}$, da je $x \in B \subseteq A$.
    \item Vzemimo preslikavo $f: (X, \mathcal{T}) \to (X', \mathcal{T}')$, kjer je $\mathcal{B}$ baza $\mathcal{T}$ in $\mathcal{B}'$ baza $\mathcal{T}'$.
    Potem je $f$ zvezna natanko tedaj, ko je $f^{-1} (\mathcal{B}') \subseteq \mathcal{T}$, in odprta natanko tedaj, ko je $f(\mathcal{B}) \subseteq \mathcal{T}'$.
  \end{enumerate}
\end{trditev}

\begin{dokaz}
  Dokažimo drugo točko trditve in sicer trditvi v levo.
  Naj bo $U \in \mathcal{T'}$ odprta množica, torej unija bazičnih množic.
  Potem je $f^{-1} (U)$ enaka uniji praslik množic, torej je odprta.
  Odprtost preslikave je prav tako enostavno pokazati, saj slika $f$ ohranja unije.
\end{dokaz}

\begin{zgled}
  Funkcija $f: S^1 \to S^1$ s predpisom $f(z) = z^n$, $n \in \Z \setminus \{0\}$ je odprta.
  Za bazo topologije vzamemo odprte loke, ki so krajši od $\frac{2 \pi}{n}$.
  Slika takega loka je odprt lok. Torej je po točki $(2)$ $f$ odprta preslikava.
\end{zgled}

\begin{zgled}
  Za bazo evklidske topologije v $\R^2$ običajno vzamemo odprte krogle.
  Projekcija odprte krogle v $\R^2$ na $\R$ je odprti interval, kar pomeni, da je projekcija odprta preslikava.
\end{zgled}

\begin{zgled}
  Oglejmo si osnovni lastnosti lokalne baze okolic.
  \begin{itemize}
    \item Krogle s središčem v $x$ so lokalna baza za $x$ v metrični topologiji 
    (lahko vzamemo tudi samo majhne krogle ali samo krogle z racionalnim polmerom).
    \item Unija lokalnih baz za vse $x \in X$ je baza celotne topologije.
    Če je $\mathcal{B}$ baza $\mathcal{T}$, potem je $\mathcal{B}_x = \{B \in \mathcal{B}\ |\ x \in B\}$
    lokalna baza okoli $x$.
  \end{itemize}
\end{zgled}

Naj bo $\mathcal{B}$ poljubna družina podmnožic $X$ in $\mathcal{T}$ množica vseh poljubnih unij elementov $\mathcal{B}$.
Ali je $\mathcal{T}$ topologija na $X$? Preverimo kriterije in hitro ugotovimo, da če je $\mathcal{B}$ pokritje $X$,
potem $\topo{T}$ ustreza prvima dvema pogojema za topologijo.
Če pa je $\mathcal{B}$ še zaprta za končne preseke, potem $\mathcal{T}$ ustreza tudi tretji točki za topologijo.
Izkaže pa se, da je ta pogoj nepotrebno oster, saj na primer ni izpolnjen pri metričnih kroglah.
Dovolj je, če je vsak $B_\lambda \cap B_\mu$
unija elementov $\mathcal{B}$, kjer sta $B_\lambda, B_\mu \in \topo{B}$.

\begin{trditev}
  Naj bo $\mathcal{B}$ družina podmnožic $X$ in $\mathcal{T}$ množica poljubnih unij elementov $\mathcal{B}$.
  Potem je $\mathcal{T}$ topologija natanko tedaj, ko je:
  \begin{itemize}
    \item $\mathcal{B}$ pokritje $X$,
    \item za vsaka $B_1, B_2 \in \mathcal{B}$ in za vsak $x \in B_1 \cap B_2$ obstaja $B \in \mathcal{B}$, da je $x \in B \subseteq B_1 \cap B_2$.
  \end{itemize}
\end{trditev}

\begin{zgled}
  Produktna topologija: naj bosta $(X, \mathcal{T})$ in $(Y, \mathcal{T}')$ topološka prostora.
  Potem množica $\{U \times V\ |\ U \in \mathcal{T},\ V \in \mathcal{T}'\}$ ni topologija (unija "`škatlastih"' množic ni nujno sama "`škatlasta"'),
  vendar pa je zaprta za končne preseke in zato tvori bazo neke topologije.
  Tej pravimo produktna topologija.
\end{zgled}

\begin{trditev}
  Projekciji $\mathrm{pr}_X: X \times Y \to X$ in $\mathrm{pr}_Y : X \times Y \to Y$ sta zvezni in odprti preslikavi.
\end{trditev}

\begin{dokaz}
  Naj bosta $(X, \mathcal{T})$ in $(Y, \mathcal{T}')$
  topološka prostora ter $X \times Y$ topološki prostor s produktno topologijo. Definiramo projekciji $\mathrm{pr}_X : X \times Y \to X$ in $\mathrm{pr}_Y : X \times Y \to Y$.
  Naj bo $U \subseteq X$. Potem je $\mathrm{pr}_X ^{-1} (U) = U \times Y$ škatlasta, torej odprta in je $\mathrm{pr}_X$ zvezna.
  Naj bo še $V \subseteq Y$. Potem je $\mathrm{pr}_X (U \times V) = U$ in dobimo, da je $\mathrm{pr}_X$ tudi odprta.
  Enako velja tudi za $\mathrm{pr}_Y$.
\end{dokaz}

\begin{zgled}
  Pokažemo lahko tudi, da projekcije v splošnem niso zaprte.
  Naj bosta $X = \R_+$, $Y = \R_+$ in označimo graf funkcije $f(x) = \frac{1}{x}$ z $\Gamma$.
  Potem je $\Gamma$ zaprt v produktu, a $\mathrm{pr}_Y (\Gamma) = (0, +\infty)$ ni zaprt v $[0, +\infty)$.
\end{zgled}

Naj bo $\mathcal{P}$ neka družina podmnožic $X$.
Kaj je minimalna topologija na $X$, v kateri so množice iz $\mathcal{P}$ odprte?
Ta topologija bi morala vsebovati vse unije in vse končne preseke elementov $\mathcal{P}$.
Izkaže se, da je to v bistvu že dovolj. Družina vseh končnih presekov elementov $\mathcal{P}$
je očitno zaprta za končne preseke, torej rabimo le še zahtevati, da je pokritje.

\begin{definicija}
  Če je $\topo{P}$ pokritje za $X$, potem družina vseh končnih presekov množic iz $\mathcal{P}$
  ustreza pogojem za bazo topologije. Družina $\mathcal{T}$ vseh množic, ki jih lahko zapišemo kot unije končnih presekov
  iz $\topo{P}$, je najmanjša topologija na $X$, ki vsebuje vse množice $\topo{P}$.
  Pravimo, da je $\topo{P}$ predbaza za topologijo $\topo{T}$.
\end{definicija}

\begin{trditev}
  Naj bo preslikava $f: (X, \mathcal{T}) \to (Y, \mathcal{T}')$ in $\mathcal{P}'$ predbaza za $\mathcal{T}'$.
  Potem je $f$ zvezna natanko tedaj, ko je $f^{-1} (\mathcal{P}') \subseteq \mathcal{T}$.
\end{trditev}

Ta trditev velja, ker $f^{-1}$ ohranja unije in preseke.
Podoben test za odprtost $f$ pa ne deluje, ker je v splošnem $f(A \cap B) \neq f(A) \cap f(B)$.

\begin{zgled}
  "`Pasovi"' so predbaza produktne topologije, saj njihovi končni preseki ("`škatlaste množice"') tvorijo 
  bazo produktne topologije. Tako je produktna topologija najmanjša topologija na produktu, za katero so projekcije zvezne.
  S predbazo praviloma vpeljemo topologijo tudi na produktih neskončno mnogo topoloških prostorov.
\end{zgled}

Vsako funkcijo $f: X \to Y_1 \times \cdots \times Y_n$ lahko napišemo po komponentah kot $f (x) = (f_1 (x), \dots, f_n (x))$.
Če je funkcija $f$ zvezna, so zvezne tudi komponente, saj jih lahko napišemo kot kompozitum s projekcijo $f_k = \mathrm{pr}_k \circ f$.
Iz prejšnje trditve pa dobimo tudi obrat.

\begin{trditev}
  Preslikava $f: Z \to X_1 \times \dots \times X_n$ je zvezna natanko tedaj, ko so vse komponente $f$ zvezne.
\end{trditev}

\begin{dokaz}
  Denimo, da so komponentne funkcije $f = (f_1, \dots, f_n)$ zvezne.
  To pomeni, da je za poljubno predbazično množico oziroma pas $\mathrm{pr}_k ^{-1} (U)$ praslika $f^{-1} (\mathrm{pr}_k ^{-1} (U)) = f_k ^{-1} (U)$ odprta,
  torej je tudi $f$ zvezna.
\end{dokaz}

Poleg definiranja topologije in ugotavljanja lastnosti funkcij imajo 
baze še eno pomembno vlogo: lahko jih uporabimo za grobo oceno velikosti topološkega prostora 
in bogastvo njegove topologije.
Tako vzamemo kardinalnost baz je mera velikosti topologije.
Če ima $\mathcal{T}$ končno bazo, je $\mathcal{T}$ končna.

\begin{definicija}
  Topološki prostor $(X, \mathcal{T})$ ustreza \textbf{1.~aksiomu števnosti} (je "`$1$-števen"'),
  če za vsak $x \in X$ obstaja števna lokalna baza okolic pri $x$.
\end{definicija}

Očitno je, da je vsak metrični prostor $1$-števen, saj lahko za lokalno bazo vzamemo krogle z racionalnim polmerom.

\begin{trditev}
  Naj bo $X$ $1$-števen. Potem veljata naslednji izjavi.
  \begin{itemize}
    \item Za vsak $A \subseteq X$ je $\overline{A} = L(A) := \{x\ |\ \text{$x$ je limita zaporedja točk v $A$}\}$.
    \item Preslikava $f: X \to Y$ je zvezna natanko tedaj, ko za vsak $A \subseteq X$ velja $f(L(A)) \subseteq L(f(A))$.
  \end{itemize}
\end{trditev}

\begin{dokaz}
  Dokažimo prvo točko. Očitno je, da velja $A \subseteq L(A) \subseteq \overline{A}$.
  Dokazati moramo, da velja tudi $\overline{A} \subseteq L(A)$.
  Za $x \in \overline{A}$ obstaja lokalna baza okolic $U_1, U_2, \dots$
  Za vsak $i \in \N$ izberemo $a_i \in A \cap U_1 \cap \dots \cap U_i$ (z indukcijo lahko pokažemo, da je slednja množica neprazna).
  Potem vsaka okolica $U$ točke $x$ vsebuje neko bazično okolico $U_i$,
  torej $U$ vsebuje $a_i, a_{i+1}, a_{i + 2}, \dots$ Od tod pa sledi, da je $x$ limita zaporedja $\{a_i\}$.
\end{dokaz}

\begin{definicija}
  Prostor zadošča \textbf{2.~aksiomu števnosti} (je "`$2$-števen"'), če obstaja števna baza za njegovo topologijo.
\end{definicija}

\begin{zgled}
  Če v $\R$ vzamemo družino vseh intervalov z racionalnimi krajišči, dobimo bazo njegove topologije, torej je $\R$ $2$-števen.
  Podobno se prepričamo, da je tudi $\R^n$ $2$-števen prostor: za bazo lahko vzamemo produkte intervalov z racionalnimi krajišči ali pa odprte krogle 
  s polmerom $\frac{1}{n}$ in središčem v točkah z racionalnimi koordinatami.
\end{zgled}

Vsak $2$-števen prostor je seveda tudi $1$-števen, a ne velja obratno. Protiprimer je recimo nešteven prostor, opremljen z diskretno metriko.
Ta je metričen (torej $1$-števen), vendar pa ni $2$-števen.

\begin{definicija}
  Podmnožica $A$ je povsod gosta v $X$, če seka vsako odprto množico $X$, ali ekvivalentno, če je $\overline{A} = X.$
  Če prostor premore kako povsod gosto podmnožico, ki je števna, pravimo, da je separabilen. 
\end{definicija}

\begin{zgled}
  Oglejmo si nekaj zgledov separabilnih prostorov.
  \begin{itemize}
    \item Evklidski prostor $\R^n$ je separabilen, saj je $\Q^n$ gosta v $\R^n$.
    \item Če je prostor $2$-števen, potem števno povsod gosto podmnožico dobimo tako, da izberemo po eno točko iz vsake bazične okolice.
    \item Prostor funkcij $\topo{C} ([a, b])$ je separabilen. Weierstrassov izrek nam pove, da lahko vsako funkcijo na $[a, b]$ enakomerno aproksimiramo s polinomi.
    Torej so polinomi gosti v $\topo{C} ([a, b])$ z običajno metriko in zato so tudi polinomi z racionalnimi koeficienti gosti v istem prostoru.
  \end{itemize}
\end{zgled}

\begin{zgled}
  V neskončnem prostoru $X$ s topologijo končnih komplementov je vsaka neskončna podmnožica 
  $A \subseteq X$ gosta, saj je $X$ edina zaprta neskončna množica in je torej lahko le $\overline{A} = X$.
\end{zgled}

Očitno nam $2$-števnost implicira separabilnost, v metričnih prostorih pa sta ta pojma celo ekvivalentna.

\begin{izrek}
  Metrični prostor $(X, d)$ je $2$-števen natanko takrat, ko v njem obstaja števna povsod gosta podmnožica.
\end{izrek}

\begin{dokaz}
  Dokazali bomo, da je za vsako povsod gosto podmnožico $A \subseteq X$ družina 
  $\mathcal{B} = \{K(x, q)\ |\ x \in A,\ q \in \Q_+\}$ baza topologije prostora $X$.
  Poljubna odprta okolica $U$ točke $x \in X$ vsebuje neko kroglo $K(x, r)$, krogla $K \left(x, \frac{r}{3}\right)$
  pa vsebuje točko $a \in A$. Potem vzamemo $q \in \left(\frac{r}{3}, \frac{r}{2}\right) \cap \Q$ in imamo 
  $x \in K(a, q) \subseteq K\left(a, \frac{r}{2}\right) \subseteq K(x, r) \subseteq U$.
  Torej je $U$ unija množic iz $\topo{B}$, torej je le-ta res baza topologije $X$.
  Če je množica $A$ števna, je števna tudi $\topo{B}$.
\end{dokaz}

V splošnem pa separabilnost (in celo hkrati separabilnost ter $1$-števnost) ne implicira $2$-števnosti.

\subsection{Podprostori}

Poljubno podmnožico metričnega prostora lahko opremimo z metriko tako, da funkcijo razdalje preprosto zožimo na točke podmnožice.
Podobno ravnamo pri topologijah. Naj bo $(X, \topo{T})$ topološki prostor.
Za poljubno podmnožico $A\subseteq X$ definiramo $\topo{T}_A = \{A \cap U\ |\ U \in \topo{T}\}$.
Tej topologiji pravimo podedovana (ali inducirana) topologija, za prostor $(A, \topo{T}_A)$ pa,
da je podprostor prostora $(X, \topo{T})$.

\begin{zgled}
  Navedimo nekaj primerov.
  \begin{itemize}
    \item Evklidska topologija na $\R^2$ inducira evklidsko topologijo na $\R \stackrel{\sim}{=} \R \times \{0\}.$
    \item Topologija, ki jo $\N$ podeduje od $\R$, je diskretna. Dovolj je pokazati, da so enoelementarne podmnožice
    $\N$ odprte v podedovani topologiji, kar je očitno.
    \item Podprostor diskretnega prostora je tudi diskreten, podprostor prostora s trivialno topologijo pa ima 
    prav tako trivialno topologijo.
  \end{itemize}
\end{zgled}

V metričnem prostoru $(X, d)$ lahko podmnožica $A \subseteq X$ podeduje topologijo tako, da se metrika $d$ zoži na $d_A$ in 
nato inducira topologijo $\topo{T}_{d_A}$, ali pa da metrika $d$ inducira topologijo $\topo{T}_d$ na $X$, ki jo potem $A$ podeduje kot topologijo $(\topo{T}_d)_A$.
Očitno velja $\topo{T}_{(d_A)} = (\topo{T}_d)_A$.

Velja tudi, da če je $B \subseteq A \subseteq X$, potem je topologija, ki jo $B$ podeduje od $X$, ista kot topologija,
ki jo $B$ podeduje od topologije, ki jo je $A$ podedoval od $X$.

\begin{trditev}
  Naj bo $(X, \topo{T})$ topološki prostor in $A \subseteq X$.
  Zaprte množice v $A$ so preseki $A$ z zaprtimi množicami v $X$.
\end{trditev}

\begin{dokaz}
  Množica $B$ je zaprta v $A$, če je $A \setminus B$ odprta v $A$, torej če je $A \setminus B = A \cap U$
  za neko odprto množico $U$ v prostoru $X$.
  Tedaj je $B = A \setminus (A \setminus B) = (A \cap X) \setminus (A \cap U) = A \cap (X \setminus U)$, torej presek $A$ z množico $X \setminus U$,
  ki je zaprta v $X$. Obratno, če je $F$ zaprta v $X$ potem je $\stcomp{F}$ odprt v $X$,
  $A \cap \stcomp{F}$ odprta v $A$ in $A \cap F = A \setminus (A \cap \stcomp{F})$ zaprta v $A$.
\end{dokaz}

\begin{trditev}
  Naj bo $\mathcal{B}$ baza $\topo{T}$. Potem je $\mathcal{B}_A = \{B \cap A\ |\ B \in \topo{B}\}$ baza $\topo{T}_A$.
  Če ima $(X, \topo{T})$ števno bazo, jo ima tudi $(A, \topo{T}_A)$.
\end{trditev}

Topološka lastnost je dedna, če se prenaša na podprostore: tedaj iz predpostavke, da ima $\topo{T}$ lastnost $L$ sledi,
da ima $\topo{T}_A$ lastnost $L$ za vsak $A \subseteq X$.

\begin{zgled}
  Oglejmo si nekaj topoloških lastnosti, ki smo jih že spoznali, in obravnavajmo njihovo dednost.
  \begin{itemize}
    \item Diskretnost in trivialnost sta dedni.
    \item $1$-števnost in $2$-števnost sta dedni.
    \item Metrizabilnost je dedna.
    \item Separabilnost je dedna le na odprtih podprostorih.
    \item Povezanost očitno ni dedna, kot tudi kompaktnost (ta je dedna le na zaprtih podprostorih -- spomni se izreka iz analize 1).
  \end{itemize}
\end{zgled}

\begin{zgled}
  Dokažimo, da separabilnost v splošnem ni dedna.
  Naj bo $X$ nešteven in diskreten (torej ni separabilen).
  Sedaj definirajmo množico $X' = X \cup \{a\}$ in okolica $a$ je samo cela množica $X'$.
  Tedaj je $X'$ separabilen, podprostor $X$ pa ne.
\end{zgled}

\begin{trditev}
  Naj bo $(X, \mathcal{T})$ in $(A, \mathcal{T}_A)$ njegov podprostor.
  Potem za $B \subseteq A$ velja $\cl_A\, {B} = A \cap \cl_X \, B$, $\inte_A\, B\supseteq A \cap \inte_X \, B$ 
  in $\fr_A \, B \subseteq A \cap \fr_X \, B$.
\end{trditev}

\begin{dokaz}
  Pokažimo, da je $\cl_A \, B = A \cap \cl_X \, B$.
  Naj bo $a \in \cl_A \, B$ in $U$ odprta množica v $X$, ki vsebuje $a$.
  Potem je $U \cap B = (A \cap U) \cap B \neq \emptyset$, torej je $a \in \cl_X\, B$.
  Obratno: naj bo $a \in \cl_X \, B$ in naj bo $a \in A \cap U$, kjer je $U$ odprta v $X$.
  Tedaj je $(A \cap U) \cap B = U \cap B \neq \emptyset$, kar pomeni, da je $a \in \cl_A\, B$.
\end{dokaz}

Množica, ki je odprta v podprostoru, ni nujno odprta v celem prostoru:
na primer množica $A$, ki ni odprta v $X$, je vendarle odprta v sami sebi.

\begin{trditev}
  Če je $B$ odprta podmnožica $A$ in $A$ odprta podmnožica $X$, potem je $B$ tudi odprta v $X$.
  Podobno velja za zaprtost.
\end{trditev}

Kakšen učinek pa ima zožitev domene ali kodomene različnih preslikav?
Vzemimo na primer inkluzijo $A$ v $X$, ki jo označimo kot $i: (A, \topo{T}_A) \to (X, \topo{T})$.
Za poljubno odprto množico $U \subseteq X$ je $i^{-1} (U) = A \cap U$,
kar pomeni, da je funkcija $i: (A, \topo{T}_A) \to (X, \topo{T})$ zvezna (in $\topo{T}_A$ je najmanjša topologija na $A$,
za katero je inkluzija zvezna).
Za poljubno zvezno preslikavo $f: X \to Y$ in podprostor $A \subseteq X$
je tudi zožitev $f \big|_A : A \to Y$ zvezna, saj jo lahko zapišemo kot kompozitum zveznih funkcij: $f \big|_A = f \circ i$.
Prav tako za $f(X) \subseteq B \subseteq Y$ velja, da je $f: (X, \topo{T}) \to (B, \topo{T}'_B)$ zvezna.

Funkcije pogosto definiramo odsekoma. Pri splošnih topoloških prostorih lahko podamo zvezne predpise na vseh množicah nekega pokritja
in poskrbimo, da se definicije na presekih ujemajo. Naj bo $\{X_\lambda\}$ pokritje (atlas) $X$.
Za družino funkcij $f_\lambda: X_\lambda \to Y$ bomo rekli, da je usklajena, če je 
$f_\lambda \big|_{X_\lambda \cap X_\mu} = f_\mu \big|_{X_\lambda \cap X_\mu}$ za poljubna indeksa $\lambda, \mu$.
Vsaka usklajena družina enolično določa funkcijo $f: X \to Y$, zanima pa nas pri kakšnih pogojih bo $f$ gotovo zvezna.

\begin{definicija}
  Pokritje $\{X_\lambda\}$ je lokalno končno, če ima vsaka točka $X$ neko okolico, ki seka le končno mnogo različnih $X_\lambda$.
\end{definicija}

\begin{zgled}
  Oglejmo si nekaj primerov lokalno končnih pokritij.
  \begin{itemize}
    \item Končna pokritja so tudi lokalno končna.
    \item $\{[n, n+1]\ |\ n \in \Z\}$ je lokalno končno pokritje za $\R$.
    \item $\{(n, \infty)\ |\ n \in \Z\}$ je lokalno končno pokritje za $(1, \infty)$, 
    kljub temu, da se vsi členi medebojno sekajo.
    \item Pokritje $\R$ s točkami ni lokalno končno, kljub temu, da so njegovi členi disjunktni.
  \end{itemize}
\end{zgled}

\begin{trditev}
  Naj bo $\{X_\lambda\}$ odprto pokritje $X$.
  Potem je $A \subseteq X$ odprt natanko tedaj, ko je $A \cap X_\lambda$ odprt v $X_\lambda$ za vse $\lambda$.
\end{trditev}

Pri zaprtih pokritjih je položaj bolj zapleten, ker analogna trditev ni pravilna.
Če $\R$ na primer pokrijemo s točkami, ima interval $(0, 1)$ zaprt presek z vsako množico
pokritja, čeprav ni zaprt v $\R$.

\begin{trditev}
  Naj bo $\{X_\lambda\}$ zaprto lokalno končno pokritje $X$.
  Potem je $A \subseteq X$ zaprt natanko tedaj, ko je $A \cap X_\lambda$ zaprt v $X_\lambda$ za vse $\lambda$.
\end{trditev}

\begin{dokaz}
  Če je $A$ zaprt, so zaprti tudi vsi preseki $X_\lambda \cap A$.
  Obratno, poljuben $x \notin A$ ima okolico $U$, ki seka le končno mnogo 
  členov pokritja, recimo $X_{\lambda_1}, \dots, X_{\lambda_n}$.
  Za vsak $i = 1, \dots, n$ je $X_{\lambda_i} \cap A$ zaprt v $X_{\lambda_i}$
  in zato tudi v $X$.
  Tedaj je $U \cap \stcomp{(X_{\lambda_1} \cap A)} \cap \dots \cap \stcomp{(X_{\lambda_n} \cap A)}$ okolica $x$, ki ne seka $A$,
  kar pomeni, da je $A$ zaprt v $X$.
\end{dokaz}

\begin{izrek}
  Naj bo $\{X_\lambda\}$ pokritje za $X$, ki je bodisi odprto bodisi lokalno končno in zaprto.
  Tedaj vsaka usklajena družina zveznih preslikav $f_\lambda: X_\lambda \to Y$ enolično določa zvezno preslikavo
  $f: X \to Y$, za katero je $f \big|_{X_\lambda} = f_\lambda$.
\end{izrek}

\begin{dokaz}
  Preverimo zveznost.
  Če je pokritje $\{X_\lambda\}$ odprto, potem za odprto množico $U \subseteq Y$ velja 
  $X_\lambda \cap f^{-1} (U) = (f \big|_{X_\lambda})^{-1} (U)$ in ta množica je odprta v $X_\lambda$
  in po prejšnji trditvi je $f^{-1} (U)$ odprta v $X$. Če pa je pokritje $\{X_\lambda\}$
  zaprto in lokalno končno, potem za zaprto množico $F \subseteq X$ sledi, da je $X_\lambda \cap f^{-1} (F) = (f \big|_{X_\lambda})^{-1} (F)$
  zaprta v $X_\lambda$ in po prejšnji trditvi je $f^{-1} (F)$ zaprta v $X$.
\end{dokaz}

\begin{posledica}
  Naj bo $\{X_\lambda\}$ pokritje $X$, ki je bodisi odprto bodisi lokalno končno in zaprto.
  Tedaj je funkcija $f: X \to Y$ zvezna natanko tedaj, ko so zvezne vse zožitve 
  $f \big|_{X_\lambda}$.
\end{posledica}

\begin{zgled}
  Naj bo $X = \{x_0, x_1, x_2, \dots\}$ števna množica, opremljena z diskretno topologijo.
  Če vsakemu $x_k$ priredimo število $k \in \R$, smo $X$ izenačili z množico $\N_0$.
  Vemo, da je topologija, ki jo naravna števila podedujejo od $\R$, diskretna, torej se topologija $X$ ujema s podedovano.
  Lahko pa točki $x_0$ priredimo $0$, ostalim $x_k$ pa vrednosti $\frac{1}{k}$.
  To pomeni, da $X$ enačimo s podmnožico $Y = \left\lbrace\frac{1}{k}\ \Big|\ k \in \N \right\rbrace \cup \{0\}$.
  Topologija, ki jo pa $Y$ podeduje od $\R$, pa ni diskretna, saj $0$ ni odprta.
\end{zgled}

\begin{definicija}
  Preslikava $f: X \to Y$ je vložitev, če je $f$ homeomorfizem med $X$ in $f(X)$ (glede na od $Y$ podedovano topologijo).
\end{definicija}

Vsaka vložitev je injektivna in zvezna, vendar pa zvezne injekcije niso nujno vložitve.
Primer je na primer prejšnji zgled. Naslednja trditev sledi takoj.

\begin{trditev}
  Naj bo $f: X \to Y$ zvezna injektivna preslikava.
  Če je $f(X)$ odprta v $Y$, potem je $f$ vložitev natanko tedaj, ko je preslikava $f: X \to Y$ odprta.
  Če pa je $f(X)$ zaprta v $Y$, potem je $f$ vložitev natanko tedaj, ko je preslikava $f: X \to Y$ zaprta.
\end{trditev}

Iz invariance odprtih množic sledi, da za odprto množico $U \subseteq \R^n$ in zvezno injektivno
funkcijo $f: U \to \R^n$ velja, da je $f$ vložitev.

\begin{zgled}
  \begin{itemize}
    \item Vsaka izometrija $f: (X, d) \to (X', d')$ je vložitev.
    \item Preslikava, ki odprti interval "`navije"' na osmico, ni vložitev.
    \item Eksponentna preslikava $g: (-\pi, \pi) \to \C$ je vložitev.
    \item Za zaprto omejeno $A \subseteq \R^n$ je vsaka preslikava $f: A \to \R^m$ zaprta,
    torej je vsaka injektivna zvezna preslikava iz $A$ v $\R^m$ vložitev.
  \end{itemize}
\end{zgled}
\clearpage
\section{Topološke lastnosti}

Pomembni matematični izreki imajo pogosto topološko utemeljitev.
Navedimo nekaj primerov.

\begin{zgled}
  \textbf{Zaporedje ne more imeti več kot ene limite} (Hausdorffova lastnost).
  Denimo, da smo v metričnem prostoru $(X, d)$. +
  Če sta $a, b \in X$ in $a \neq b$ limiti zaporedja, potem se tako v vsakem odprtem krogu $K_1$ s središčem v $a$,
  kot tudi v vsakem odprtem krogu $K_2$ s središčem v $b$ nahajajo skoraj vsi členi našega zaporedja.
  Če pa polmera krogov $K_1$ in $K_2$ ustrezno zamnjšamo, da sta disjunktna, pridemo v protislovje.
\end{zgled}

\begin{zgled}
  \textbf{Vsako omejeno zaporedje v $\R^n$ vsebuje konvergentno podzaporedje} (Bolzano-Weierstrass).
  Ta izrek izvira iz dejstva, da je zaprtje omejene množice kompaktno (v $\R^n$, ne v splošnem).
\end{zgled}

\begin{zgled}
  \textbf{Presek padajočega zaporedja zaprtih krogel je neprazen} (Cantor).
  To sledi iz dejstva, da so zaprte krogle kompaktne.
\end{zgled}

\begin{zgled}
  \textbf{Metrični prostor, ki je poln in brez izoliranih točk, je nujno nešteven}.
  Točka $x \in X$ je izolirana, če ima okolico, ki ne seka $X \setminus \{x\}$.
  To velja natanko tedaj, ko je $\{x\}$ odprt podprostor $X$.
  Dejstvu, da števen metrični prostor brez izoliranih točk ne more biti poln,
  pravimo Bairov izrek.
\end{zgled}

V tem poglavju bomo obravnavali tri sklope topoloških lastnosti: ločljivost, povezanost in kompaktnost.

\subsection{Ločljivost}

\begin{definicija}
  Topologija $\topo{T}$ na množici $X$ loči disjunktni množici $A, B \subseteq X$, 
  če obstajata množici $U, V \in \topo{T}$, da je $A \subseteq U$,
  $B \cap U = \emptyset$ in $B \subseteq V$, $A \cap V = \emptyset$.
\end{definicija}

\begin{definicija}
  Topologija $\topo{T}$ ostro loči disjunktni množici $A$ in $B$, če obstajata
  množici $U, V \in \topo{T}$, da velja $A \subseteq U$, $B \subseteq V$ in $U \cap V = \emptyset$.
\end{definicija}

\begin{zgled}
  Oglejmo si nekaj primerov ločljivosti topologij.
  \begin{itemize}
    \item Trivialna topologija ne loči ničesar, medtem ko diskretna topologija ostro loči vse disjunktne podmnožice.
    \item Topologija končnih komplementov loči vsak par različnih točk, a jih ne loči ostro 
    (razen v primeru, ko je $X$ končna množica -- takrat topologija končnih komplementov seveda sovpada z diskretno topologijo.)
    \item Oglejmo si točke v zaprtju množice $A$. Če je $x \in \overline{A} \setminus A$,
    potem topologija po definiciji ne loči točke $x$ od množice $A$.
    Od tod sledi, da je smiselno gledati le, kako topologija loči zaprte množice.
  \end{itemize}
\end{zgled}

\begin{definicija}
  Prostor $(X, \topo{T})$ je Hausdorffov, če topologija $\topo{T}$ 
  ostro loči vsaki dve različni točki $X$.
\end{definicija}

\begin{zgled}
  Videli smo že, da je vsak metrični prostor Hausdorffov, topologija končnih 
  komplementov na neskončni množici pa ne.
\end{zgled}

\begin{trditev}
  Naslednje izjave so ekvivalentne.
  \begin{itemize}
    \item Prostor $X$ je Hausdorffov.
    \item Za poljuben $x \in X$ je $\displaystyle \bigcap_{U \in \mathcal{U}} \overline{U} = \{x\}$, kjer je $\mathcal{U}$ družina vseh okolic $x$.
    \item Diagonala $\Delta_X = \{(x, x) \in X \times X\ |\ x \in X\}$ je zaprt podprostor $X \times X$.
  \end{itemize}
\end{trditev}

\begin{dokaz}
  $(1) \Rightarrow (2)$ Naj bo prostor Hausdorffov in vzemimo $x \neq y$.
  Tedaj obstajata okolici $U$ za $x$ in $V$ za $y$, ki sta disjunktni.
  Sedaj uporabimo dejstvo, da $y \notin \overline{U}$, zato ni v preseku zaprtij vseh okolic točke $x$.

  $(2) \Rightarrow (3)$ Izberimo točko $(x, y)$, ki ni v $\Delta$.
  Naj bo $U$ odprta okolica $x$, katere zaprtje ne vsebuje $y$.
  Potem je $U \times \stcomp{\overline{U}}$ okolica točke $(x, y)$, ki ne seka diagonale.

  $(3) \Rightarrow (1)$ Izberimo $x \neq y$. Potem $(x, y) \notin \Delta$, zato ima 
  $(x, y)$ škatlasto okolico $U \times V$, ki ne seka diagonale, kar pomeni, 
  da $U \cap V = \emptyset$. Odprti množici $U$ in $V$ ostro ločita $x$ in $y$.
\end{dokaz}

Navedimo nekaj pomembnih lastnosti Hausdorffovih prostorov.

\begin{izrek}
  Naj bo $Y$ Hausdorffov.
  \begin{enumerate}
    \item Vsaka končna podmnožica $Y$ je zaprta.
    \item Točka $y$ je stekališče množice $A \subseteq Y$ natanko takrat, ko vsaka okolica $y$ vsebuje neskončno točk iz $A$.
    \item Zaporedje v $Y$ ima največ eno limito.
    \item Za zvezni funkciji $f, g: X \to Y$ je množica točk ujemanja $\{x \in X\ |\ f(x) = g(x)\}$ zaprta v $X$.
    \item Če se preslikavi $f, g: X \to Y$ ujemata na neki gosti podmnožici $X$, potem je $f = g$.
    \item Graf preslikave $f: X \to Y$ je zaprt podprostor produkta $X \times Y$.
  \end{enumerate}
\end{izrek}

\begin{dokaz}
  $(4)$ Funkcija $(f, g): X \to Y \times Y$ je zvezna. 
  Množica točk ujemanja je zaprta, saj je enaka prasliki 
  diagonale v $Y \times Y$, ki pa je zaprta po prejšnji trditvi.

  $(6)$ Definirajmo preslikavi $u, v: X \times Y \to Y$ kot $u(x, y) = f(x)$ in $v(x, y) = y$.
  Množica točk ujemanja $u$ in $v$ je ravno graf $\Gamma_f = \{(x, y) \in X \times Y\ |\ y = f(x)\} \subseteq X \times Y$
  funkcije $f$, ki je po $(4)$ zaprt.
\end{dokaz}

\begin{izrek}
  Naj bo prostor $X$ $1$-števen, prostor $Y$ pa Hausdorffov.
  Potem je funkcija $f: X \to Y$ zvezna natanko tedaj, ko za vsako konvergentno 
  zaporedje $(x_n)$ v $X$ velja $\lim f(x_n) = f (\lim x_n)$.
\end{izrek}

\begin{dokaz}
  $(\Rightarrow)$ Naj bo $f$ zvezna in naj zaporedje $(x_n)$ v $X$ konvergira proti $x$.
  Za poljubno okolico $U$ točke $f(x)$ je zaradi zveznosti $f^{-1} (U)$ okolica točke $x$,
  kar pomeni, da so v $f^{-1} (U)$ skoraj vsi elementi zaporedja $(x_n)$,
  torej so v $U$ skoraj vsi elementi zaporedja $(f(x_n))$, torej je $\lim f(x_n) = f(\lim x_n)$.

  $(\Leftarrow)$ Naj bo $x$ točka, ki je v zaprtju množice $A \subseteq X$ in naj bo 
  $U_1, U_2, \dots$ neka baza okolic za $x$. Tedaj za vsak $n$ lahko izberemo točko $x_n \in A \cap U_1 \cap \dots \cap U_n$.
  Tako smo konstruirali zaporedje $(x_n)$, ki konvergira proti $x$ in so vsi $f(x_n) \in f(A)$.
  Po predpostavki je $f(x) = f(\lim x_n) = \lim f(x_n) \in \overline{f(A)}$.
  To pa pomeni, da za poljuben $A \subseteq X$ velja $f(\overline{A}) \subseteq \overline{f(A)}$,
  torej je $f$ zvezna.
\end{dokaz}

\begin{definicija}
  Prostor $(X, \topo{T})$ je Fréchetov, če $\topo{T}$ vsako točko $X$ loči od vsake druge točke $X$.
\end{definicija}

\begin{zgled}
  Vsi Hausdorffovi prostori so seveda tudi Fréchetovi, prostor s trivialno topologijo 
  pa je primer prostora, ki ni Fréchetov. Neskončen prostor s topologijo končnega komplementa pa je Fréchetov prostor,
  ki ni Hausdorffov.
\end{zgled}

\begin{trditev}
  Prostor $X$ je Fréchetov natanko tedaj, ko so vsi enojčki zaprti.
\end{trditev}

Iz trditve sledi, da so v Fréchetovem prostoru vse končne množice zaprte in vsi 
komplementi končnih množic odprti. Torej je topologija Fréchetova natanko tedaj, ko vsebuje 
topologijo končnih komplementov.

\begin{trditev}
  Hausdorffova in Fréchetova lastnost sta dedni in multiplikativni.
\end{trditev}

\begin{dokaz}
  Dokažimo to za Fréchetovo lastnost. Naj bo $X$ Fréchetov in $A \subseteq X$.
  Naj bosta $a, a' \in A$. Ker je $X$ Fréchetov, obstajata odprti množici $U, V \subseteq X$, ki
  ločita točki $a$ in $a'$, torej sta $A \cap U$ in $A \cap V$ odprti množici v $A$ ločita ti dve točki.
  Podobno naredimo za multiplikativnost: naj bosta $X$ in $Y$ Fréchetova in naj bosta $(x, y)$ in $(x', y')$
  različni točki v produktu $X \times Y$. Brez škode za splošnost predpostavimo, da je $x \neq x'$ in zaradi 
  Fréchetove lastnosti obstajata odprti disjunktni množici $U, V \subseteq X$, ki ločita $x$ in $x'$.
  Potem pa odprti množici $U \times Y$ in $V \times Y$ ločita točki $(x, y)$ in $(x', y')$.
\end{dokaz}

\begin{definicija}
  Prostor $(X, \topo{T})$ je regularen, če je Fréchetov in $\topo{T}$
  ostro loči točke od zaprtih množic.
\end{definicija}

\begin{definicija}
  Prostor $(X, \topo{T})$ je normalen, če je Fréchetov in $\topo{T}$
  ostro loči (disjunktne) zaprte množice.
\end{definicija}

Takoj lahko vidimo, da normalnost implicira regularnost, ta implicira Hausdorffovo lastnost,
ki pa nazadnje implicira Fréchetovo lastnost.
Implikacij se ne da obrniti: denimo, da je $\topo{T}$ normalna.
Če je $\topo{T} \subseteq \topo{T}'$, je tudi $\topo{T}'$ Hausdorffova,
ni pa nujno regularna, ker imamo tudi nove zaprte množice, ki se jih ne da nujno ostro ločiti od točk.

\begin{zgled}
  Vzemimo evklidsko topologijo $\topo{T}$ na $\R$.
  Naj bo $\topo{T}'$ najmanjša topologija, ki vsebuje $\topo{T}$ in $\{\Q\}$.
  Baza za to topologijo je torej $\{(a, b), (a, b) \cap \Q\}$.
  Potem je $\topo{T}'$ Hausdorffova, vendar pa ne moremo ostro ločiti točke $0$ od $\stcomp{\Q}$,
  saj je $\R$ edina odprta množica, ki vsebuje $\stcomp{\Q}$.
\end{zgled}

\begin{izrek}
  Vsak metričen prostor je normalen.
\end{izrek}

\begin{dokaz}
  Za disjunktni zaprti množici $A, B \subseteq X$ definiramo $U = \{x \in X\ |\ d(A, x) < d(B, x)\}$
  in $V = \{x \in X\ |\ d(A, x) > d(B, x) \}$. Očitno je $A \subseteq U$, $B \subseteq V$ in $U \cap V = \emptyset$,
  dokazati moramo le še, da sta $U$ in $V$ res odprti. Za poljuben $x \in U$ definiramo $r = \frac{1}{2}(d(B, x) - d(A, x))$
  in potem je $K(x, r) \subseteq U$. Tako se lahko prepričamo, da sta $U$ in $V$ res odprti.
\end{dokaz}

\begin{trditev}
  Zaprt podprostor normalnega prostora je normalen.
\end{trditev}

\begin{dokaz}
  Naj bo $A$ zaprt podprostor normalnega prostora $X$.
  Če sta $B$ in $C$ disjunktna zaprta podprostora v $A$, sta disjunktna 
  in zaprta tudi v $X$. Zaradi normalnosti obstajata odprti množici $U$ in $V$,
  ki ostro ločita $B$ in $C$.
  Potem pa sta $U \cap A$ in $V \cap A$ v $A$ odprti množici, ki seveda ostro ločita $B$ in $C$.
\end{dokaz}

Kaj pa se zgodi, če $A$ ni zaprt?
V tem primeru sta $B$ in $C$ oblike $B = B' \cap A$ 
in $C = C' \cap A$, kjer sta $B'$ in $C'$ zaprti podmnožici $X$.
Nimamo pa nobenega zagotovila, da sta $B'$ in $C'$ disjunktni,
zato ne moremo uporabiti normalnosti $X$.
V posebnem primeru, ko je $B = \{b\}$ enoelementna množica, lahko 
vzamemo $B' = \{b\}$, torej sta $B'$ in $C'$ disjunktni.
Po normalnosti dobimo odprti množici $U$ in $V$, ki ju ostro ločita, zato $U \cap A$ 
in $V \cap A$ ostro ločita $b$ in $C$. Sklepamo torej, da je vsak podprostor normalnega prostora 
regularen. Podobno pokažemo,
da je podprostor regularnega prostora tudi regularen, torej je regularnost 
dedna (in pa celo produktna) lastnost.

\begin{trditev}
  Vsak podprostor regularnega prostora je regularen.
\end{trditev}

Primeri regularnih prostorov, ki niso normalni in primeri, ki kažejo, da normalnost ni dedna 
oziroma multiplikativna lastnost, so komplicirani, kar nakazuje naslednji izrek.

\begin{izrek}[Tihonov]
  Prostor, ki je regularen in 2-števen je normalen.
\end{izrek}

\begin{dokaz}
  Naj bodsta $A$ in $B$ zaprta podprostora regularnega $2$-števnega prostora $X$. 
  Zaradi regularnosti $X$ lahko za vsako točko $x \in A$ najdemo bazično okolico 
  $U$, ki ne seka neke okolice $B$, kar pomeni, da njeno zaprtje $\overline{U}$ ne 
  seka $B$. Če to naredimo za vse točke $A$, dobimo števno pokritje $\{U_1, U_2, \dots\}$
  za $A$ z odprtimi množicami, katerih zaprtja ne sekajo $B$. Na enak način dobimo 
  števno pokritje $\{V_1, V_2, \dots\}$ za $B$ z odprtimi množicami, katerih zaprtja ne sekajo $A$.
  Sedaj za vsak $n \in \N$ definiramo $U_n ' = U_n \setminus \bigcup_{i = 1} ^n \overline{V_i}$
  in $V_n ' = V_n \setminus \bigcup_{i = 1} ^n \overline{U_i}$. Sedaj je $U' = \bigcup_i U'_i$ okolica $A$,
  $\bigcup_i V'_i$ okolica $B$ in očitno je $U' \cap V' = \emptyset$.
\end{dokaz}

\begin{definicija}
  Različne stopnje ločljivosti je mogoče sistematično predstaviti kot zaporedje vedno ostrejših zahtev.
  Temu zaporedju pravimo aksiomi ločljivosti.
  \begin{itemize}
    \item $X$ je $T_0$: Za različni točki $x,x' \in X$ obstaja okolica ene izmed teh dveh točk, ki jo loči od druge.
    \item $X$ je $T_1$: Za različni točki $x,x' \in X$ obstaja okolica točke $x$, ki jo loči od $x'$ in okolica $x'$, ki jo loči od $x$.
    \item $X$ je $T_2$: Za različni točki $x,x' \in X$ obstajata okolici, ki ostro ločita $x$ in $x'$.
    \item $X$ je $T_3$: Za točko $x\in X$ in zaprto množico $A \subseteq X$, ki ne vsebuje $x$, obstajata okolici, ki ostro ločita $x$ in $A$.
    \item $X$ je $T_4$: Za disjunktni zaprti množici $A, B \subseteq X$ obstajata okolici, ki ostro ločita $A$ in $B$.
  \end{itemize}
\end{definicija}

Zahteva $T_0$ je minimalna zahteva, da topologija sploh loči točke.
Zahtevi $T_1$ in $T_2$ sta zaporedno enaki Fréchetovi oziroma Hausdorffovi lastnosti.
Iz definicije pa nam zahteva $T_3$ nam v kombinaciji s $T_1$ da regularnost, zahteva $T_4$ 
pa skupaj s $T_1$ da normalnost. Hitro vidimo, da veljata naslednji ekvivalenci.

\begin{trditev}\label{trd:1}
  Prostor $X$ ima lastnost $T_3$ natanko tedaj, ko za vsak $x \in X$ in vsako odprto okolico $U$
  za $x$ obstaja taka odprta množica $V$, da velja $x \in V \subseteq \overline{V} \subseteq U$.
\end{trditev}

\begin{trditev}
  Prostor $X$ ima lastnost $T_4$ natanko tedaj, ko za vsako zaprto podmnožico $A \subseteq X$ in vsako 
  odprto okolico $U$ za $A$ obstaja taka odprta množica $V$, da velja $A \subseteq V \subseteq \overline{V} \subseteq U$.
\end{trditev}

Iz trditve \ref{trd:1} pa takoj sledi naslednja trditev.

\begin{trditev}
  Produkt regularnih prostorov je regularen.
\end{trditev}

\begin{dokaz}
  Ker vemo, da je Fréchetova lastnost multiplikativna, zadošča pokazati,
  da je produkt prostorov $X$ in $X'$ z lastnostjo $T_3$ tudi $T_3$.
  Naj bo $W$ okolica $(x, x') \in X \times X'$. Potem $W$ po definiciji vsebuje neko škatlasto 
  okolico $U \times U'$ točke $(x, x')$. Potem ima $x$ okolico $V$, da je $x \in V \subseteq \overline{V} \subseteq U$
  in $x'$ okolico $V'$, da je $x' \in V' \subseteq \overline{V'} \subseteq U'$.
  Potem pa je $x \in \overline{V} \times \overline{V'} \subseteq U \times U'$ in $X \times X'$ je $T_3$.
\end{dokaz}

\subsection{Povezanost}

\begin{definicija}
  Prostor $(X, \topo{T})$ je nepovezan, če ga lahko razcepimo kot disjunktno unijo dveh nepraznih, odprtih množic: 
  $X = U \cup V$, $U, V \in \topo{T} \setminus\{ \emptyset \}$ in $U \cap V = \emptyset$.
  Obratno je $(X, \topo{T})$ povezan, če ni nepovezan.
\end{definicija}

\begin{trditev}
  Naslednje izjave so ekvivalentne:
  \begin{itemize}
    \item $X$ je nepovezan.
    \item $X$ lahko predstavimo kot disjunktno unijo dveh nepraznih, zaprtih množic.
    \item V $X$ obstaja prava, neprazna podmnožica, ki je odprta in zaprta.
    \item Obstaja zvezna surjekcija $f: X \to \{0, 1\}$ (z diskretno topologijo).
  \end{itemize}
\end{trditev}

\begin{dokaz}
  Dokažimo ekvivalenco 1. in 4. točke.
  Če obstaja razcep prostora $X$ na neprazni disjunktni odprti množici $U$ 
  in $V$, potem lahko definiramo $f$, ki slika $U$ v $0$ in $V$ v $1$.
  Obratno: če obstaja taka zvezna surjekcija $f$, potem imamo razcep $X = f^{-1} (0) \cup f^{-1} (1)$, saj sta ti množici zaradi surjektivnosti neprazni.
\end{dokaz}

\begin{izrek}
  Množica $A \subseteq \R$ je povezana natanko tedaj, ko je interval.
\end{izrek}

\begin{dokaz}
  Denimo, da $A$ ni interval. Potem obstajajo točke $a < b< c$,
  tako da sta $a, c \in A$ in $b \notin A$. Potem imamo razcep $A$ na množici $A \cap (- \infty, b)$ 
  in $A \cap (b, \infty)$. 

  Obrat: denimo, da je $A$ interval, ki ni povezan. Potem imamo razcep $A = U \cup V$,
  kjer sta $U, V$ odprti in neprazni. Naj bosta $a \in U$, $c \in V$ in brez škode 
  za splošnost predpostavimo $a \leq c$.
  Potem definiramo $b = \sup \{x \in \R\ |\ [a, x) \subseteq U\}$. 
  Očitno je $a \leq b \leq c$, zato je $b \in A$.
  Hkrati pa velja $b \in \cl_A\, U$, torej je $b \in U$,
  saj je $U$ tudi zaprt v $A$.
  Tedaj pa pridemo v protislovje z definicijo $b$, saj iz odprtosti $U$ sledi,
  da obstaja $\varepsilon > 0$, da $(b - \varepsilon, b + \varepsilon) \subseteq U$.
\end{dokaz}

\begin{izrek}
  Zvezna slika povezanega prostora je povezan prostor.
\end{izrek}

\begin{dokaz}
  Naj bo $f: X \to Y$ zvezna. Če $f(X)$ ni povezan, potem obstaja 
  zvezna surjekcija $g: f(X) \to \{0, 1\}$.
  Tedaj je tudi $g \circ f: X \to \{0, 1\}$ zvezna surjekcija, torej $X$ ni povezan.
\end{dokaz}

\begin{izrek}\label{izr:1}
  Veljajo naslednje trditve.
  \begin{itemize}
    \item Če so $\{A_\lambda\}$ povezane podmnožice $X$ in $\bigcap_\lambda A_\lambda \neq \emptyset$, je množica $\bigcup_\lambda A_\lambda$ povezana.
    \item Produkt povezanih prostorov je povezan.
    \item Če za $a, b \in A$ obstaja pot $\gamma: I \to A$, da je $\gamma(0) = a$ in $\gamma(1) = b$, je $A$ povezan.
    \item Če je $A \subseteq X$ povezan in $A \subseteq B \subseteq \overline{A}$, je $B$ povezan (operacija zaprtja ohranja povezanost). 
  \end{itemize}
\end{izrek}

\begin{dokaz}
  Prvo točko dokažemo tako, da predpostavimo $a \in \bigcap_\lambda U_\lambda$ in $f(a) = 0$,
  od koder sledi $f(\bigcup A_\lambda) = 0$.
  Pri drugi točki pa upoštevamo, da je produkt povezane množice z enojcem prav tako povezana množica.
  Oglejmo si dokaza tretje in četrte točke.
  Denimo, da lahko poljubni točki v $X$ povežemo s potjo in naj bo $f: X \to \{0, 1\}$ zvezna surjekcija.
  Če potem pot $\gamma$ komponiramo z $f$, dobimo, 
  da obstaja zvezna surjektivna preslikava intervala $(0, 1)$
  v $\{0, 1\}$, kar vodi v protislovje. Nazadnje dokažimo še četrto točko.
  Naj bo $A$ povezana in $A \subseteq B \subseteq \overline{A}$.
  Denimo, da obstaja zvezna preslikava $f: B \to \{0, 1\}$ in $f(A) = 0$.
  Potem je edina zvezna razširitev $f$ na točke v zaprtju kar $f(B) = 0$, zato $f$ ni surjektivna.
\end{dokaz}

\begin{zgled}
  Navedimo nekaj zgledov povezanih množic.
  \begin{itemize}
    \item Konveksne množice v $\R^n$ so povezane.
    \item Sfere $S^n$ so povezane.
    \item Komplement končne množicev $\R^n$ je povezan.
    \item Komplement $\Q^n$ v $\R^n$ je povezan.
    \item Prostora $\R$ in $\R^n$ nista homeomorfna, saj homeomorfizmi ohranjajo povezanost.
  \end{itemize}
\end{zgled}

\begin{izrek}
  Naj bo $X$ povezana in $f: X \to \R$ zvezna, potem je $f(X)$ interval.
\end{izrek}

\begin{zgled}
  Oglejmo si gref $L \subseteq \R^2$ funkcije $f(x) = \sin \frac{\pi}{x}$ za $x \in [-1, 0).$
  Prostor $L$ je povezan, saj je homeomorfen intervalu $[-1, 0)$.
  Zato je povezano tudi njegovo zaprtje $\overline{L} = L \cup (\{0\} \times [-1, 1])$.
  Vendar pa lahko pokažemo, da v $\overline{L}$ ni zvezne poti od točke $(-1, 0)$ do $(0, 0).$
  Ta množica (rečemo ji tudi Varšavski lok) nam pove, da povezanost ne implicira povezanosti s potmi.
\end{zgled}

\begin{definicija}
  Prostor $X$ je povezan s potmi, če za vsaka $a, b \in X$ obstaja pot $\gamma: (I, 0, 1) \to (X, a, b)$.
\end{definicija}

Povezanost s potmi seveda implicira povezanost. V nekaterih pomembnih primerih pojma sovpadata.

\begin{definicija}
  Za $x \in X$ definiramo $C(x)$ kot unijo vseh povezanih podmnožic $X$, ki vsebujejo $x$.
  Po izreku \ref{izr:1} je $C(x)$ oziroma komponenta $x$ maksimalna povezana množica $X$, ki vsebuje $x$.
  Komponente $X$ so maksimalne povezane podmnožice $X$.
\end{definicija}

\begin{zgled}
  V diskretnem prostoru so komponente točke. Enako velja za $\Q$,
  saj so edine povezane množice v podmnožicah $\R$ intervali.
\end{zgled}

\begin{definicija}
  Prostor $X$ je povsem nepovezan, če so njegove komponente enojčki.
\end{definicija}

\begin{izrek}
  ~ \vspace{-1em}
  \begin{itemize}
    \item Komponente tvorijo particijo prostora (torej razdelitev na disjunktne podprostore).
    \item Komponente $X$ so zaprti podprostori $X$.
    \item Naj bo $f: X \to Y$ zvezna. Potem je slika vsake komponente $X$ v celoti vsebovana v neki komponenti $Y$. 
  \end{itemize}
\end{izrek}

Zaprtost komponent sledi iz maksimalnosti: če je množica $A$ namreč povezana, je povezano tudi njeno zaprtje.
Če ima $X$ končno mnogo komponent, potem so le-te tudi odprte (zato, ker je vsaka komponenta komplement unije ostalih komponent, končna unija zaprtih množic pa je zaprta).
V splošnem pa komponente niso odprte, kar je jasno že iz primera $\Q$.

Za odprtost komponent mora vsak $x \in X$ imeti okolico $x \in U \subseteq C(x)$.
Zadoščalo bi že, če ima vsak $x$ lokalno bazo iz povezanih množic.

\begin{definicija}
  Prostor $(X, \mathcal{T})$ je lokalno povezan, če ima bazo okolic iz povezanih množic.
\end{definicija}

\begin{zgled}
  Primer lokalno povezanega prostora je $\R^n$, saj so krogle konveksne in zato povezane.
  Ker se lokalna povezanost deduje na odprte podprostore, je tudi vsaka odprta množica $U \subseteq \R^n$ lokalno povezana.
\end{zgled}

\begin{zgled}
  Diskreten prostor je lokalno povezan, ni pa povezan. Nasprotno pa je varšavski lok povezan, a ni lokalno povezan.
\end{zgled}

\begin{trditev}
  Prostor $X$ je lokalno povezan natanko tedaj, ko so komponente vsake odprte podmnožice $X$ tudi odprte.
\end{trditev}

\begin{dokaz}
  Dokaz trditve v levo je očiten, če za bazo topologije vzamemo komponente odprtih množic.
\end{dokaz}

Večino do sedaj omenjenih trditev lahko ponovimo tudi za povezanost s potmi.

\begin{definicija}
  Za $x \in X$ je $C'(x)$ unija vseh podmnožic $X$, ki so povezane s potmi in vsebujejo $x$.
\end{definicija}

Velja, da so unije s potmi povezanih podmnožic, ki imajo neprazen presek, povezane s potmi.
Tako so $C'(x)$ maksimalne s potmi povezane podmnožice $X$ oziroma potne komponente $X$.
Podobno tudi potne komponente tvorijo particijo $X$ in za zvezno funkcijo $f: X \to Y$ je $f$-slika
potne komponente $X$ povsem vsebovana v neki potni komponenti $Y$.
Vendar pa je razlika v tem, da potne komponente niso nujno zaprte -- protiprimer je ponovno varšavski lok.

\begin{definicija}
  Prostor $(X, \topo{T})$ je lokalno povezan s potmi, če ima bazo iz odprtih množic, ki so povezane s potmi.
\end{definicija}

\begin{izrek}
  V prostoru, ki je lokalno povezan s potmi, so komponente isto kot potne komponente.
\end{izrek}

\begin{dokaz}
  Vzemimo $x \in X$. Vemo, da je $C'(x) \subseteq C(x)$, zato dokazujemo še inkluzijo v obratno smer.
  Zaradi lokalne povezanosti s potmi za vsak $y \in C'(x)$ obstaja odprta s potmi povezana okolica $U \ni y$,
  da je $U \subseteq C'(y) = C'(x)$, zato je $C'(x)$ odprta v $X$ in tudi v $C(x)$.
  Obenem pa je $C'(x)$ komplement unije vseh ostalih potnih komponent $C(x)$, ki so vse odprte, zato je $C'(x)$ tudi zaprta v $C(x)$.
  Ker je $C(x)$ povezana, mora biti $C'(x) = C(x)$.
\end{dokaz}

\begin{posledica}
  Če je $X$ povezana in lokalno povezana s potmi, potem je $X$ povezan s potmi.
\end{posledica}

\begin{posledica}
  Odprta podmnožica v $\R^n$ je povezana natanko takrat, ko je povezana s potmi.
\end{posledica}

\subsection{Kompaktnost}

Mnogi pomembni izreki, ki smo jih obravnavali pri analizi, so dejansko topološki in so posledice kompaktnosti.

\begin{definicija}
  Prostor $(X, \topo{T})$ je kompakten, če ima vsako odprto pokritje končno podpokritje.
\end{definicija}

Bistveno za kompaktnost je, da iz še tako drobnega, neskončnega pokritja lahko izluščimo končno poddružino, ki je pokritje.
Kompaktnost lahko testiramo na bazi: velja, da je $X$ kompakten natanko tedaj, ko ima vsako pokritje z bazičnimi okolicami 
končno podpokritje. Pri dokazovanju kompaktnosti podprostora $A$ dovolj preveriti, 
da vsako pokritje $A$ z množicami, ki so odprte v $X$, premore končno podpokritje.

\begin{zgled}
  Oglejmo si nekaj osnovnih zgledov kompaktnih prostorov.
  \begin{itemize}
    \item Končni prostori so kompaktni.
    \item Če je $(x_n)_{n \in \N}$ konvergentno zaporedje in $\lim_{n \to \infty} x_n = a$,
    potem je $\{x_1, x_2, \dots\} \cup \{a\}$ kompakt.
    \item Množica $\R$ ni kompaktna. V splošnem je v metričnih prostorih vsak kompakt omejen.
    \item Diskreten prostor je kompakten natanko tedaj, ko je končen.
  \end{itemize}
\end{zgled}

\begin{izrek}
  Množica $[a, b]$ je kompaktna.
\end{izrek}

\begin{dokaz}
  Naj bo $\{U_\lambda\}$ pokritje $[a, b]$ z intervali. Definirajmo 
  $$c = \sup \{x \in [a, b]\ |\ \mathrm{\text{$[a, x]$ lahko pokrijemo s končno mnogo $U_\lambda$}} \}.$$
  Potem ne more biti $c < b$, saj je $c$ vsebovan v neki odprti množici $U_\lambda$ in je 
  $[a, x] \cup U_\lambda$ prav tako pokrit s končno mnogo $U_\lambda$, kar pa je protislovje z definicijo $c$.
\end{dokaz}

\begin{izrek}
  Zvezna slika kompakta je kompakt.
\end{izrek}

\begin{dokaz}
  Naj bo $f: X \to Y$ zvezna in $X$ kompakt.
  Označimo neko odprto pokritje za $f(X)$ z $\mathcal{U}$.
  Potem je $f^{-1} (\mathcal{U})$ odprto pokritje za $X$ in posledično je 
  $\{f^{-1} (U_{\lambda_1}), \dots, f^{-1} (U_{\lambda_n})\}$ pokritje za $X$,
  zato je $\{U_{\lambda_1}, \dots, U_{\lambda_n}\}$ pokritje za $f(X)$.
\end{dokaz}

\begin{izrek}
  Zaprt podprostor kompaktnega prostora je kompakten.
\end{izrek}

\begin{dokaz}
  Naj bo $\mathcal{U}$ pokritje $A$ z množicami, ki so odprte v $X$.
  $\mathcal{U} \cup \{\stcomp{A}\}$ je odprto pokritje $X$, zato obstaja končno podpokritje tega pokritja,
  ki je tudi pokritje $A$ z elementi $\mathcal{U}$.
\end{dokaz}

\begin{izrek}
  Kompaktnost je multiplikativna.
\end{izrek}

\begin{dokaz}
  Naj bo $\mathcal{U}$ pokritje $X \times Y$ z odprtimi škatlastimi okolicami.
  Vzemimo $x \in X$. Potem obstajajo take odprte množice $U_1 \times V_1, \dots, U_n \times V_n \in \mathcal{U}$,
  ki pokrivajo $\{x\} \times Y$ (ta množica je homeomorfna $Y$ in zato kompaktna).
  Definiramo $U_x = U_1 \cap \dots \cap U_n \ni x$ kot odprto okolico $x$.
  Torej ima $U_x \times Y$ odprto končno podpokritje z elementi $\mathcal{U}$.
  To naredimo za vse $x \in X$ in dobimo družimo odprtih množic $\{U_{x}\}_{x \in X}$.
  Ker je $X$ kompaktna, imamo podpokritje $\{U_{x_1}, \dots, U_{x_n}\}$ in sedaj je dokaz kompaktnosti na dlani.
\end{dokaz}

Z indukcijo seveda dobimo, da je za kompakte $X_1, \dots, X_n$  tudi prostor $X_1 \times \dots \times X_n$ kompakten.
Od tod direktno sledi, da je zaprta podmnožica produkta intervalov kompaktna.
Velja pa še veliko več: celo poljuben produkt kompaktov je kompakten (izrek Tihonova).

\begin{posledica}
  Vsaka omejena, zaprta podmnožica $\R^n$ je kompaktna.
\end{posledica}

\begin{trditev}
  Naj bo $K$ kompaktna podmnožica Hausdorffovega prostora $X$. Potem je $K$ zaprt v $X$.
\end{trditev}

\begin{dokaz}
  Za vsak $a \in K$ in $x \notin K$ obstajata odprti množici $U_a \ni a$ in $V_a \ni x$, da je $U_a \cap V_a = \emptyset$.
  Ker družina množic $\{U_a\}$ pokriva $K$, obstaja podpokritje $U_{a_1} \cup \dots \cup U_{a_n} \supseteq K$.
  Ta unija pa je disjunktna z množico $V_{a_1} \cap \dots \cap V_{a_n}$, ki je odprta okolica $x$.
  Dokazali smo še več: $T_2$ topologije ostro ločijo točke od kompaktov.
\end{dokaz}

\begin{izrek}
  Množica $A \subseteq \R^n$ je kompaktna natanko tedaj, ko je $A$ zaprta in omejena.
\end{izrek}

\begin{posledica}
  Vsaka preslikava iz kompaktnega prostora $f: X \to \R$ je omejena ter zavzame maksimum in minimum.
\end{posledica}

\begin{trditev}
  V kompaktu ima vsaka neskončna množica stekališče.
\end{trditev}

\begin{dokaz}
  Naj bo $A \subseteq X$, kjer je $X$ kompakt in $A$ neskončna.
  Denimo, da $A$ nima stekališča. Potem za vsak $x \in X$ obstaja $U_x$, ki vsebuje le končno mnogo 
  točk iz $A$. Potem je $\{U_x\}$ odprto pokritje za $X$ in obstaja podpokritje $U_{x_1}, \dots, U_{x_n}$, ki pokrivajo $X$ in $A$ je končna.
  Ker je $A$ neskončna, ima stekališče.
\end{dokaz}

\begin{izrek}[Bolzano-Weierstrass]
  Vsako omejeno zaporedje v $\R^n$ ima konvergentno podzaporedje.
\end{izrek}

Kompaktnost pa tudi okrepi nekatere druge topološke lastnosti.

\begin{izrek}
  Kompakten Hausdorffov prostor je normalen.
\end{izrek}

\begin{dokaz}
  Dokazati moramo lastnost $T_4$.
  Naj bosta $A, B$ zaprti disjunktni množici.
  Od tod sledi, da sta kompaktni.
  Ker Hausdorffov prostor loči kompakte od točk, za vsako točko $a \in A$ 
  obstajata odprti disjunktni množici $U_a \ni a$ in $V_a \supseteq B$.
  Ker je $\{U_a\}$ pokritje $A$, obstaja končno podpokritje $U_{a_1}, \dots, U_{a_n}$ in tako sta 
  $U_{a_1} \cup \dots \cup U_{a_n} \supseteq A$ in $V_{a_1} \cap \dots \cap V_{a_n} \supseteq B$ odprti disjunktni množici, ki ločita $A$ in $B$.
\end{dokaz}

\begin{izrek}
  Kompakten metrični prostor je $2$-števen (in zato tudi separabilen).
\end{izrek}

\begin{dokaz}
  Naj bo $X$ metričen kompakt. Za vsak $n \in \N$ izberemo končno 
  podpokritje pokritja $\topo{U}_n = \{K(x, \frac{1}{n})\ |\ x \in X\}$,
  na primer $K(x^{(n)}_1, \frac{1}{n}), \dots, K(x^{(n)}_{k_n}, \frac{1}{n}).$
  Če vsa ta končna podpokritja združimo, dobimo števno bazo za $X$ in $\{x_i ^{(n)}\}$
  je števna gosta podmnožica $X$.
\end{dokaz}

Vendar pa moramo upoštevati, da $1$-števnost in kompaktnost še ne implicirata $2$-števnosti.

\begin{trditev}
  Prostor $X$ je kompakten natanko takrat, ko v vsaki družini zaprtih množic s praznim presekom 
  obstaja končna poddružina, katere presek je prav tako prazen.
\end{trditev}

\begin{posledica}[Cantorjev izrek]
  Naj bo $X$ kompakten in $F_{1} \supseteq F_2 \supseteq \dots$ neprazne zaprte množice. Potem je $\bigcap F_i \neq \emptyset$.
\end{posledica}

\begin{dokaz}
  Denimo, da je $\bigcap F_i = \emptyset$.
  Potem obstajajo $F_{i_1}, \dots, F_{i_n}$, da je $F_{i_1} \cap F_{i_2} \cap \dots \cap F_{i_n} = \emptyset$.
  To pa pomeni, da je $F_{\max \{i_k\}} = \emptyset$.
\end{dokaz}

V nadaljevanju bomo pokazali, da so zvezne funkcije na metričnih kompaktih enakomerno zvezne.
Ključni vmesni korak pa je naslednja trditev.

\begin{lema}[Lebesgue]
  Za vsako odprto pokritje $\mathcal{U}$ metričnega kompakta $X$ obstaja Lebesgueovo
  število $\lambda > 0$ z lastnostjo, da je vsaka krogla z radijem manjšim od $\lambda$ vsebovama v nekem elementu $\topo{U}$.
\end{lema}

\begin{dokaz}
  Za začetek izberemo podpokritje $U_1, \dots, U_n$ pokritja $\mathcal{U}$.
  Ogledamo si $$f(x) = \max \{d(x, \stcomp{U_1}), \dots, d(x, \stcomp{U_n})\}.$$
  Takoj vidimo, da je $f$ zvezna, ker je maksimum zveznih funkcij, in pozitivna, saj je 
  $U_1, \dots, U_n$ pokritje. Sedaj definiramo $\lambda = \min_{x \in X} f(x) > 0$.
  Ta obstaja, saj je $X$ kompakt.
\end{dokaz}

\begin{posledica}
  Naj bosta $X$ in $Y$ metrična prostora in $X$ kompakt. Potem je zvezna funkcija $f: X \to Y$ tudi enakomerno zvezna.
\end{posledica}

Podobno kot druge lastnosti ima tudi kompaktnost svojo lokalizirano različico.

\begin{definicija}
  Prostor $X$ je lokalno kompakten, če ima vsaka točka $x \in X$ kakšno kompaktno okolico (torej je $x$ notranja v nekem kompaktu $K \subseteq X$).
\end{definicija}

Vsak kompakten prostor je očitno lokalno kompakten. Primer prostora, ki je lokalno kompakten, a ni kompakten, pa je kar $\R^n$.
Definicija lokalne kompaktnosti se nekoliko razlikuje od običajnega opisa lokalnih lastnosti.

\begin{definicija}
  Odprta podmnožica $U \subseteq X$ je relativno kompaktna, če je $\overline{U}$ kompakt.
\end{definicija}

\begin{trditev}
  Hausdorffov prostor je lokalno kompakten natanko tedaj, ko ima bazo iz relativno kompaktnih množic.
\end{trditev}

\begin{dokaz}
  Naj bo $x \in X$ in $U$ njegova odprta okolica.
  Po predpostavki obstaja tak kompakt $K$, da je $x$ vsebovan v njegovi 
  notranjosti. Po enemu od prejšnjih izrekov je $K$ normalen, zato je množica $K \cap U$ regularna.
  Torej obstaja odprta množica $V$, da je $x \in V \subseteq \overline{V} \subseteq K \cap U$ in ker je 
  $K$ kompakt, $\overline{V}$ pa zaprta je zato prav tako kompaktna.
\end{dokaz}

\begin{izrek}
  Vsak lokalno kompakten Hausdorffov prostor je regularen.
\end{izrek}

\begin{dokaz}
  Dovolj je dokazati lastnost $T_3$.
  Vzemimo $x \in U$, kjer je $U$ poljubna odprta podmnožica $X$.
  Po prejšnji trditvi obstaja relativno kompaktna množica $V$, da je $x \in V \subseteq U$.
  Ker je $\overline{V}$ kompakten in Hausdorffov, je normalen in zato je $V$ regularen, torej obstaja odprta množica $W$, da je 
  $x \in W \subseteq \overline{W} \subseteq V \subseteq U$.
\end{dokaz}

Lokalno kompaktni prostori imajo Bairovo lastnost.

\begin{definicija}
  Prostor $X$ je Bairov, če ga ni mogoče predstaviti kot števno unijo zaprtih množic s prazno notranjostjo.
\end{definicija}

\begin{zgled}
  Ravnine ni mogoče pokriti s števno mnogo gladkih krivulj.
\end{zgled}

Metrični prostor je poln, če ima vsako Cauchyjevo zaporedje limito v prostoru.
Polni metrični prostori so Bairovi.
Lokalna kompaktnost je torej posplošitev polnosti.

\begin{izrek}
  Naj bodo $F_1, F_2, \dots$ zaprte s prazno notranjostjo v lokalno kompaktnem Hausdorffovem prostoru $X$.
  Potem ima tudi $F_1 \cup F_2 \cup \dots$ prazno notranjost v $X$.
\end{izrek}

\begin{dokaz}
  Dokazati moramo, da nobena odprta podmnožica $X$ ni vsebovana v $\bigcup_{i = 1} ^\infty F_i.$
  Naj bo $U_0 \subseteq X$ odprta.
  Ker je $\inte\, F_1 = \emptyset$, $U_0$ ni podmnožica $F_1$, zato obstaja $x_1 \in U_0 \setminus F_1$.
  Potem obstaja relativno kompaktna okolica $x_1 \in U_1 \subseteq U_0$, ki ne seka $F_1$.
  Sedaj vzamemo $x_2 \in U_1 \setminus F_2$. Ponovno obstaja relativno kompaktna množica $U_2$, da je $\overline{U_2} \subseteq \overline{U_1}$.
  To počnemo v neskončnost in dobimo množice $$\overline{U_1} \supseteq \overline{U_2} \supseteq \cdots, \quad \bigcap \overline{U}_i \neq \emptyset,$$
  torej obstaja $x \in U_0 \setminus \bigcup_{i=1}^\infty F_i$.
\end{dokaz}

\begin{izrek}
  Naj bodo $F_1, F_2, \dots$ zaprte s prazno notranjostjo v polnem metričnem prostoru $X$.
  Potem ima tudi $F_1 \cup F_2 \cup \cdots$ prazno notranjost v $X$.
\end{izrek}

\begin{zgled}
  Vemo, da zvezna funkcija iz $I = [0, 1]$ v $\R$ ni nujno odvedljiva v vseh točkah, 
  vendar pa se izkaže, da ne rabi biti odvedljiva v prav nobeni točki $I$. Še več, takšne so skoraj
  vse zvezne funkcije. Naj bo $N_k$ množica vseh preslikav $f \in \mathcal{C} (I)$,
  za katere obstaja vsaj ena točka $x \in I$, blizu katere so vsi desni diferenčni količniki 
  manjši od $k$:
  $$N_k = \left\lbrace f \in \mathcal{C} (I)\ \Big|\ \exists x \in [0, 1 - \frac{1}{k}],\ \forall h \in [0, \frac{1}{k}]:\ \left| \frac{f(x+h) - f(x)}{h} \right| \leq k \right\rbrace.$$
  Vsaka preslikava, ki je odvedljiva v vsaj eni točki, je vsebovana v nekem $N_k$, zato je dovolj utemeljiti, da je 
  $\mathcal{C}(I) \setminus \bigcup_k N_k \neq \emptyset.$ Vemo, da je $\mathcal{C} (I)$ poln metrični prostor glede na 
  supremum metriko, zato je dovolj pokazati, da so množice $N_k$ zaprte s prazno notranjostjo (to lahko pokažemo s preslikavo lomljenko).
  Od tod po Bairu sledi, da nikjer odvedljive zvezne preslikave obstajajo in so goste v prostoru vseh preslikav $\mathcal{C} (I)$.
\end{zgled}

Včasih je koristno, če lahko dani prostor $X$ gledamo kot podprostor kompaktnega prostora.

\begin{definicija}
  Kompaktifikacija prostora $X$ je gosta vložitev $h : X \to \hat{X}$, kjer je $\hat{X}$ kompakten Hausdorffov prostor.
\end{definicija}

Kadar je vložitev $h$ znana, pravimo kompaktifikacija kar prostoru $\hat{X}$. Kompakten Hausdorffov prostor je sam svoja 
(edina) kompaktifikacija. Naravno vprašanje je, katere prostore sploh lahko kompaktificiramo. 
Zahtevati moramo vsaj, da je začetni prostor regularen, če pa želimo, da je odprto vložen v svoji kompaktifikaciji 
(tako da na primer dodamo le končno mnogo točk), pa mora biti 
tudi lokalno kompakten. Naj bo $X$ torej lokalno kompakten regularen prostor, ki mu dodamo $\infty$ oziroma točko v neskončnosti.
Na $X^+ = X \cup \{\infty\}$ za odprte razglasimo množice, ki ne vsebujejo $\infty$ in ki so odprte v $X$,
ter množice, ki vsebujejo $\infty$ in katerih komplementi so kompaktni v $X$. Družino takih množic označimo s $\mathcal{T}$.

\begin{izrek}
  $(X^+, \mathcal{T})$ je kompakten prostor. Če $X$ ni kompakten, je inkluzija $i: X \hookrightarrow X^+$ kompaktifikacija, ki ji pravimo 
  kompaktifikacija z eno točko ali kompaktifikacija Aleksandrova.
\end{izrek}

Pokazati se da, da je kompaktifikacija Aleksandrova enolično določena s prostorom $X$.


\clearpage
\section{Prostori preslikav}

\subsection{Topologije na prostorih preslikav}

Naj bosta $X, Y$ topološka prostora in $C(X, Y)$ množica vseh zveznih preslikav $f: X \to Y$.
V $C(X, Y)$ želimo vpeljati pojme topologije, konvergence oziroma metrike.

\begin{zgled}
  Prostor $C([a, b]) = C([a, b], \R)$ je metrični prostor za $d(f, g) = \max_{x \in [a, b]} |f(x) -g(x)|$.
  Na $C(\R)$ to ne deluje, ker je lahko množica $\{|f(x) - g(x)|\ |\ x \in \R\}$ neomejena.
  Zato pa lahko na prostoru omejenih preslikav $C_b (\R)$ definiramo metriko $d(f, g) = \sup_{x \in \R} |f(x) - g(x)|$.
  Pri tem naša intuicija sloni na tem, da sta si $f$ in $g$ blizu, če sta njuna grafa blizu v navpični smeri. 
\end{zgled}

Za okolico $f: X \to Y$ vzamemo vse funkcije, ki so blizu $f$ na $A \subseteq X$.
Za $A \subseteq X$ in $U \subseteq Y$ definiramo $\sprod{A}{U} = \{f\ |\ f(A) \subseteq U\}$.
Za množico $U$ običajno vzamemo odprte množice v $Y$, za $A$ pa imamo nekaj standardnih izbir, kot na primer 
točke, kompakti, zaprte množice ali kar cel $X$.
Za vsako od teh izbir vzamemo minimalno topologijo, ki jih vsebuje (torej topologijo podamo s predbazo).

\begin{trditev}
  Naj bo $\topo{T}_p$ topologija na $C(X, Y)$, ki jo definira predbaza $\{\sprod{x}{U}\ |\ x \in X,\ U \subseteq Y\}$.
  Naj bo $f_1, f_2, \dots$ zaporedje funkcij v $C(X, Y)$. Tedaj družina funkcij $\{f_i\}$ 
  konvergira k $f$ v topologiji $\topo{T}_p$ natanko takrat, ko je $\lim_{i \to \infty} f_i (x) = f(x)$ za $\forall x \in X$.
\end{trditev}

\begin{opomba}
  To je natanko definicija konvergence funkcij po točkah.
\end{opomba}

\begin{dokaz}
  Dokažimo trditev v levo $(\Leftarrow)$. Naj $\{f_i\}$ po točkah
  konvergira k $f$ in naj bo $f$ vsebovana v neki bazni množici $\sprod{x_1}{U_1} \cap \dots \cap \sprod{x_k}{U_k}$.
  Potem velja $f(x_1) \in U_1$ in obstaja $N_1 \in \N$, da za $i > N_1$ velja $f_i (x_1) \in U_1$.
  Naprej: $f(x_2) \in U_2$ in obstaja $N_2 \in \N$, da je $N_2 > N_1$ in za $i > N_2 $ velja $f_i (x_2) \in U_2$.
  To nadaljujemo, dokler ne dobimo, da obstaja $N_k \in \N$, da je $N_k > \dots > N_2 > N_1$ in za $i > N_k$ velja $f_i (x_k) \in U_k$.
  Torej imamo tako naravno število $N_k$, da za vsak $i > N_k$ velja $f_i \in \sprod{x_1}{U_1} \cap \dots \cap \sprod{x_k}{U_k}$.
\end{dokaz}

\begin{opomba}
  Topologijo $\topo{T}_p$ je topologija konvergence po točkah. Pri $\topo{T}_p$ topologija $X$ ne igra nobene vloge 
  (oziroma je kot če bi bil prostor $X$ diskreten).
\end{opomba}

Naj bodo $A$ kompakti. Tedaj za predbazo vzamemo množico $$\{\sprod{K}{U}\ |\ K \subseteq X\ \mathrm{\text{kompaktna}},\ U \subseteq Y\ \mathrm{\text{odprta}}\}.$$
Tipična bazična množica je torej $\sprod{K_1}{U_1} \cap \dots \cap \sprod{K_n}{U_n}$.
Omenimo še, da velja $\bigcap_{i = 1} ^n \sprod{K_i}{U} = \sprod{\bigcup_{i = 1} ^n K_i}{U}$ in 
$\bigcap_{i = 1} ^n \sprod{K}{U_i} = \sprod{K}{\bigcap_{i = 1} ^n U_i}$.
Takšni topologiji pravimo $\topo{T}_{co}$ oziroma kompaktno-odprta topologija.
Izkaže se, da za kompakten $X$ in metričen $Y$ ta topologija sovpada s topologijo enakomerne konvergence.

\begin{trditev}
  Če je $Y$ metričen, potem za bazo $\topo{T}_{co}$ na $C(X, Y)$ lahko vzamemo 
  $$\mathcal{B} = \{\left\langle f, K, \varepsilon\right\rangle\ |\ f: X \to Y,\ K \subseteq X\ \mathrm{\text{kompaktna}},\ \varepsilon > 0 \},$$
  kjer je $\langle f, K, \varepsilon \rangle = \{g: X \to Y\ |\ d_y (f(x), g(x)) < \varepsilon,\ \forall x \in K\}$.
\end{trditev}

\begin{dokaz}
  Najprej dokažimo, da je $\mathcal{B}$ res baza. Očitno je pokritje, zato predpostavimo, da velja 
  $g \in \langle f_1, K_1, \varepsilon_1 \rangle \cap \langle f_2, K_2, \varepsilon_2 \rangle$.
  Sedaj proglasimo $K = K_1 \cup K_2$ in 
  $$\varepsilon = \min \{\varepsilon_1 - \max_{x \in K_1} d(f_1(x), g(x)),\ \varepsilon_2 - \max_{x \in K_2} d(f_2 (x), g(x))\}.$$
  Potem velja $g \in \langle g, K, \varepsilon \rangle \subseteq \langle f_1, K_1, \varepsilon_1 \rangle \cap \langle f_2, K_2, \varepsilon_2 \rangle$.
  Vemo, da velja $\mathcal{T}_{co} \subseteq \topo{T}_{ec}$, saj je za vsako predbazično množico $\sprod{K}{K(y, \varepsilon)} = \langle \mathrm{const}_y, K, \varepsilon \rangle$.
  Dokažimo še v nasprotno smer ($\topo{T}_{ec} \subseteq \mathcal{T}_{co}$), torej da $\langle f, K, \varepsilon \rangle$ vsebuje neko okolico iz topologije $\topo{T}_{co}$.
  Definirajmo $U_c = \left\lbrace x \in K\ |\ d(f(x), f(c)) < \frac{\varepsilon}{2}\right\rbrace$ za $c \in K$.
  Potem $\{U_c\}$ tvorijo odprto pokritje $K$, zato obstajajo $U_{c_1},\dots, U_{c_n}$, ki so pokritje $K$.
  Sedaj pa imamo 
  \begin{equation*}
    \sprod{\overline{U}_{c_1}}{K(f(c_1),\frac{\varepsilon}{2})} \cap \dots \cap \sprod{\overline{U}_{c_n}}{K(f(c_n),\frac{\varepsilon}{2})} \subseteq \langle f, K, \varepsilon \rangle \qedhere
  \end{equation*}
\end{dokaz}

Če imamo torej kompakten $X$ in metričen $Y$, se kompaktno-odprta topologija ujema s topologijo enakomerne konvergence.
Če pa je $X$ poljuben in $Y$ metričen, pa je $\topo{T}_{co}$ topologija enakomerne konvergence na kompaktih.
To pa je ravno topologija, v kateri konvergirajo Taylorjeve vrste.

Prostor $Y$ lahko enačimo s podprostorom konstantnih preslikav v $C(X, Y)$, saj je 
$i: Y \to C(X, Y)$ s predpisom $y \mapsto c_y$ vložitev.
Če $Y$ ustreza $T_2$, je ta vložitev zaprta. Od tod sledi, da če $Y$ nima neke dedne lastnosti, je tudi $C(X, Y)$ nima.
Recimo, če $Y$ ni metrizabilen, potem tudi $C(X, Y)$ ni.

\begin{izrek}
  Prostor $C(X, Y)$ s kompaktno odprto topologijo je metrizabilen natanko tedaj,
  ko je $Y$ metrizabilen in $X$ hemi-kompakten (je števna unija kompaktov).
\end{izrek}

\begin{trditev}
  Prostor $(C(X, Y), \mathcal{T}_{co})$ je Hausdorffov oziroma regularen natanko tedaj, ko je $Y$ Hausdorffov oziroma regularen.
\end{trditev}

Dotaknimo se še povezanosti in kompaktnosti v $C(X, Y)$.
Pot v prostoru $C(X, Y)$ imenujemo homotopija. To je zvezna družina zveznih funkcij iz $X$ v $Y$,
indeksirnih z $[0, 1]$, ki se začne pri $f_0$ in konča pri $f_1$.
Kompaktnost v $C(X) = C(X, \R)$ pa je vsebina izreka Arzela-Ascoli.

\begin{izrek}[Arzela-Ascoli]
  Naj bo $X$ kompakten. Potem je $F \subseteq C(X)$ kompaktna natanko tedaj,
  ko je $F$ zaprta, omejena in enako-zvezna.
\end{izrek}

\begin{opomba}
  V tem kontekstu je družina množic omejena, če velja $|f(x)| \leq M$ za $\forall x \in X$ in $\forall f \in F$.
  Družina množic je enako-zvezna\footnote{equi-continuous family}, kadar za $\forall \varepsilon > 0$ obstaja 
  $\delta > 0$, da za $\forall f \in F$ iz $d(x, x') < \delta$ sledi $|f(x) - f(x')| < \varepsilon$.
\end{opomba}

\subsection{Preslikave na normalnih prostorih}

Obravnavali bomo $C(X) = C(X, \R)$.
V splošnem ni nujno, da obstajajo nekonstantne zvezne preslikave $X \to \R$.
Takšen primer je na primer neskončen $X$ s topologijo končnih komplementov.
To velja celo za nekatere regularne prostore.
Pokazali bomo, da pa na normalnih prostorih obstaja veliko zveznih preslikav.

\begin{izrek}[Urisonova lema]
  Prostor $X$ je $T_4$ natanko tedaj, ko za vsak par zaprtih, disjunktnih 
  $A, B \subseteq X$ obstaja zvezna preslikava $f: X \to [0, 1]$,
  da je $f(A) = 0$ in $f(B) = 1$.
\end{izrek}

\begin{dokaz}
  Dokažimo trditev v desno $(\Rightarrow)$.
  Naj $T_4$ prostor $X$ vsebuje disjunktni zaprti množici $A, B$.
  Naj bo $U_1 = X \setminus B$ in potem velja $A \subseteq U_1$.
  Potem zaradi lastnosti $T_4$ obstaja odprta $U_0$, da je $A \subseteq U_0 \subseteq \overline{U_0} \subseteq U_1$.
  Naprej: obstaja odprta $U_{\frac{1}{2}}$, da je $\overline{U}_0 \subseteq U_{\frac{1}{2}} \subseteq \overline{U}_{\frac{1}{2}} \subseteq U_1$.
  To nadaljujemo v neskončnost in dobimo množico 
  $\{U_\frac{a}{2^n}\ |\ 0 \leq a \leq 2^n,\ n \geq 0\}$, kjer 
  za indeksa $r < s$ velja $A \subseteq \overline{U}_r \subseteq U_s \subseteq X \setminus B$.
  Sedaj definiramo $f: X \to [0, 1]$ s predpisom 
  $$f(x) = \begin{cases}
    \inf \{r\ |\ x \in U_r\}; & x \notin B\\
    1 ; & x \in B
  \end{cases}.$$
  Ta funkcija je zvezna, saj je za $x$ v prasliki $(0, 1)$ odprta okolica 
  kar $U_s \setminus \overline{U_r}$ (natančnejša ideja dokaza je na predavanjih).
\end{dokaz}

\begin{posledica}
  Če je prostor $X$ normalen, potem za $x \neq x'$ v $X$
  obstaja $f: X \to I$, $f(x) \neq f(x')$.
  Pravimo, da realne funkcije ločijo točke.
\end{posledica}

Če je $X$ metričen, lahko Urisonovo funkcijo zapišemo s pomočjo metrike:
$f(x) = \frac{d(x, A)}{d(x, A) + d(x, B)}$.
Pri tem je $A = f^{-1} (0)$ in $B = f^{-1} (1)$, kar je močnejše kot za splošne normalne prostore.
Temu pravimo, da je $X$ popolnoma normalen.
V obratni smeri pa lahko Urisonove funkcije uporabimo za konstrukcijo metrike.
Če je $X$ normalen z neko bazo $\topo{B}$,
potem za $x \neq x'$ v $X$ lahko najdemo $B, B' \in \topo{B}$,
da je $x \in B \subseteq \overline{B} \subseteq B'$ in $x' \notin B'$.
Za vsak tak par lahko izberemo Urisonovo funkcijo $f_{B, B'} : (X, \overline{B}, X \setminus B') \to (I, 0, 1)$,
ki slika $x$ v $0$ in $x'$ v $1$.
Torej, če je $\mathcal{B}$ števna, potem obstaja števna družina funkcij v $C(X, [0, 1])$, ki loči točke $X$.

\begin{izrek}
  Vsak normalen, $2$-števen prostor je metrizabilen.
\end{izrek}

\begin{dokaz}
  Normalen, $2$-števen prostor $X$ bomo vložili v prostor kvadratno sumabilnih realnih zaporedij z 
  metriko $d\left((x_n), (y_n)\right) = \sqrt{\sum_{n = 1} ^\infty (x_n - y_n)^2}$.
  Običajna oznaka za ta prostor je $l^2$.
  Naj bo sedaj $\{f_n: X \to [0, 1]\ |\ n \in \N\}$ števna družina Urisonovih funkcij, ki loči točke $X$.
  Definiramo preslikavo $f: X \to l^2$ s predpisom $f(x) = \left(\frac{1}{n} f_n (x)\right)_{n \in \N}$.
  Ta preslikava je dobro definirana, saj je za vsak $x$ zaporedje $\left(\frac{1}{n} f_n(x)\right)$
  kvadratno sumabilno.

  Nadalje dokažimo zveznost $f$. Naj bo $x \in X$, $\varepsilon > 0$ in dovolj velik $N$,
  da velja $\sum_{n > N} \frac{1}{n^2} < \frac{\varepsilon^2}{2}$.
  Za $n = 1, \dots, N$ so množice $U_n = \left\lbrace y \in X\ |\ (f_n (x) - f_n (y))^2 < \frac{\varepsilon^2}{2N} \right\rbrace$ 
  odprte v $X$, zato je odprt tudi njihov presek $U = \bigcap_{n = 1} ^N U_n$.
  Za poljuben $y \in U$ je 
  \begin{align*}
    d(f(x), f(y))^2 &= \sum_{n = 1} ^\infty \frac{(f_n (x) - f_n (y))^2}{n^2}\\
    &= \sum_{n = 1} ^N \frac{(f_n (x) - f_n (y))^2}{n^2} + \sum_{n >N} \frac{(f_n (x) - f_n (y))^2}{n^2}\\
    &< N \frac{\varepsilon^2}{2N} + \frac{\varepsilon^2}{2} = \varepsilon^2,
  \end{align*}
  torej je $U$ okolica točke $x$, ki se z $f$ preslika v $\varepsilon$-okolico točke $f(x)$,
  s čimer pa smo pokazali zveznost.

  Dokažimo še zveznost inverza $f^{-1} : f(X) \to X$.
  Naj bo $x \in X$ in naj bo $U$ poljubna okolica $x$.
  Obstajata bazični okolici $B$ in $B'$, za kateri velja $x \in B \subseteq \overline{B} \subseteq B'$
  in naj bo $f_n$ Urisonova preslikava, prirejena paru $\overline{B},\ U \setminus B'$.
  Če za $y \in X$ velja $d(f(x), f(y)) < \frac{1}{n}$, potem iz definicije razdalje $d$ sledi 
  $|f_n (y)| < |f_n (x) - f_n (y)| < n d(f(x), f(y)) < 1$, torej je $y \in B' \subseteq U$.
  Od tod sledi, da za vse $f(y)$, ki so dovolj blizu $f(x)$, točka $y = f^{-1} (f(y))$ leži v $U$,
  zaradi česar je $f$ zvezna.
\end{dokaz}

Če prejšnji izrek združimo z izrekom Tihonova, dobimo naslednjo trditev.

\begin{izrek}
  Vsak regularen, $2$-števen prostor je metrizabilen.
\end{izrek}

\begin{opomba}
  Seveda obstajajo tudi metrizabilni prostori, ki niso $2$-števni, kot na primer 
  neštevna množica z diskretno topologijo.
  Znani so izreki, ki točno karakterizirajo metrizabilnost (izrek Nagata-Smirnov).
\end{opomba}

Pri Urisonovi lemi lahko $[0, 1]$ nadomestimo s poljubnim intervalom $[a, b]$ (preprosto komponiramo z raztegom $[0, 1]$ na $[a, b]$).
Lemo lahko razumemo kot razširitev stopničaste funkcije na $A \cup B$ do zvezne funkcije na celem $X$.
To bomo močno posplošili do izreka, ki zagotavlja razširitev poljubne zvezne preslikave iz zaprte množice $A \subseteq X$
na cel $X$.

\begin{izrek}[Tietzejev razširitveni izrek]
  Naj bo $A$ zaprta podmnožica normalnega prostora $X$.
  Potem lahko vsako zvezno preslikavo $f: A \to J$ razširimo do zvezne preslikave 
  $F: X \to J$, kjer je $J$ interval.
\end{izrek}

Izreki o razširitvah so pomembni, na primer Hahn-Banachov izrek, izrek o analitičnem nadaljevanju in tako dalje.

\begin{opomba}
  Prostor $X$ je povezan s potmi natanko tedaj, ko vsako zvezno 
  preslikavo $f: \{0, 1\} \to X$ lahko razširimo do zvezne preslikave 
  $F: [0, 1] \to X$.
\end{opomba}

Urisonova lema je v resnici zgolj zelo poseben primer Tietzejevega izreka.

\begin{lema}
  Naj bo $A$ zaprta podmnožica normalnega prostora $X$ in $f: A \to [-c, c]$ zvezna.
  Potem obstaja zvezna $h: X \to \left[-\frac{c}{3}, \frac{c}{3}\right]$,
  za katero je $|f(a) - h(a)| \leq \frac{2}{3} c$ za vse $a \in A$.
\end{lema}

\begin{dokaz}
  Naj bo $A_+ = \left\lbrace a \in A\ |\ f(x) \geq \frac{c}{3}\right\rbrace$ in $A_- = \left\lbrace a \in A\ |\ f(x) \leq -\frac{c}{3}\right\rbrace$. 
  Potem za $h$ vzamemo Urisonovo funkcijo $h: (X, A_{-}, A_+) \to \left([-\frac{c}{3}, \frac{c}{3}], -\frac{c}{3}, \frac{c}{3}\right)$.
\end{dokaz}

\begin{dokaz}[Dokaz izreka]
  Brez škode za splošnost privzamemo $J = [-1, 1]$.
  Funkcijo $F$ dobimo z vsoto $F = h_1 + h_2 + \dots$.
  Začnemo z $f: A \to [-1, 1]$, uporabimo lemo in dobimo $h_1: X \to \left[-\frac{1}{3}, \frac{1}{3}\right]$.
  Naprej: uporabimo lemo na funkciji $f - h_1: A \to \left[-\frac{2}{3}, \frac{2}{3}\right]$,
  in dobimo $h_2 : X \to \left[-\frac{2}{9}, \frac{2}{9}\right]$.
  Ta postopek nadaljujemo s funkcijo $f - h_1 - h_2: A \to \left[-\left(\frac{2}{3}\right)^2, \left(\frac{2}{3}\right)^2\right]$.
  Funkcija $h_1 + h_2 + \dots$ je zvezna po Weierstrassovem kriteriju in velja $F\Big|_A = f$.
\end{dokaz}

Po Tietzejevemu izreku lahko vsako preslikavo iz zaprtega podprostora normalnega 
prostora v realna števila razširimo do preslikave, definirane na celem prostoru.
Ta pomembna razširitvena lastnost ni značilna samo za prostor $\R$,
temveč tudi za $\R^n$, saj lahko razširimo vsako komponento posebej.

\begin{definicija}
  Prostor $E$ je absolutni ekstenzor (za normalne prostore),
  če lahko vsako preslikavo $f: A \to E$ iz zaprtega podprostora $A$ normalnega prostora $X$
  razširimo do preslikave, ki je definirana na celem $X$.
\end{definicija}

Očitno je produkt absolutnih ekstenzorjev absolutni ekstenzor.

\begin{definicija}
  Podprostor $A \subseteq X$ je retrakt prostora $X$, če obstaja preslikava 
  $r: X \to A$, imenovana retrakcija, z lastnostjo $r(a) = a$ za vse $a \in A$ 
  (je torej razširitev identične preslikave $A$ nase).
\end{definicija}

\begin{zgled}
  Oglejmo si nekaj primerov vpeljanih pojmov.
  \begin{itemize}
    \item Vsaka točka v prostoru je retrakt celega prostora: retrakcija je kar konstantna preslikava.
    Nasprotno pa par točk ne more biti retrakt povezanega prostora.
    \item Rezina $x_0 \times Y$ je retrakt $X \times Y$. Retrakcija je dana s projekcijo $r(x, y) = (x_0, y)$.
    \item Krogla $B^n$ je retrakt prostora $\R^n$, kjer je retrakcija dana s predpisom $r(x) = \frac{x}{\max\{\|x\|, 1\}}.$
    Sfera $S^{n - 1}$ je retrakt $B^n - 0^n$ in $\R^n - 0^n$ z retrakcijo $r(x) = \frac{x}{\|x\|}$,
    vendar pa po Brouwerjevem izreku $S^{n - 1}$ ni retrakt $B^n.$
  \end{itemize}
\end{zgled}

Naj bo $r: X \to A$ retrakcija, $i : A \hookrightarrow X$ pa inkluzija.
Torej je $A$ ravno množica točk ujemanja identitete in preslikave $i \circ r$.
Torej je retrakt Hausdorffovega prostora vedno zaprt podprostor.

\begin{trditev}
  Retrakt absolutnega ekstenzorja je absolutni ekstenzor.
\end{trditev}

\begin{definicija}
  Nosilec preslikave $f: X \to \R$ definiramo kot $\overline{f^{-1} (\R \setminus \{0\})}$.
  Potem pa je razčlenitev enote, podrejena nekemu pokritju $\{U_1, \dots, U_n\}$ prostora $X$,
  družina preslikav $\rho_i: X \to I$
  za katere je nosilec $\rho_i$ vsebovan v $U_i$ za vsak $i = 1, \dots, n$ in velja 
  $\rho_1 (x) + \dots + \rho_n (x) = 1$ za vse $x \in X$.
\end{definicija}

\begin{izrek}
  Za vsako odprto pokritje $\{U_1, \dots, U_n\}$ normalnega prostora $X$ 
  obstaja neka podrejena razčlenitev enote.
\end{izrek}

\subsection{Stone-Weierstrassov izrek}

Znano je, da lahko vsako zvezno funkcijo aproksimiramo s polinomom.
Natančneje, za vsako zvezno $f: [a, b] \to \R$ in vsak $\varepsilon > 0$ obstaja polinom $p \in \R[x]$,
za katerega je $|f(x) - p(x)| < \varepsilon$ za vse $x \in [a, b]$.
Standardni dokaz je konstruktiven, z uporabo tako imenovanih Bernsteinovih polinomov na intervalu $[0, 1]$.

\begin{zgled}
  Označimo $\binom{n}{i} (1-x)^{n - i} x^i = B_{n, i} (x)$ kot Bernsteinovo bazo.
Na primer:
\begin{itemize}
  \item $B_{4, 0} = (1-x)^4$,
  \item $B_{4, 1} (x) = 4 (1 - x)^3x$,
  \item $B_{4, 2} = 6 (1 - x^3)x^2$,
  \item $B_{4, 3} = 4(1 - x)x^3$,
  \item $B_{4, 4} = x^4$.
\end{itemize}  
Za fiksen $n$ so polinomi $B_{n, i}$ razčlenitev enote na $[0, 1]$:
$B_{n, 0} + B_{n, 1} + \dots + B_{n, n} (x) = 1.$ 
\end{zgled}

Za zvezno funkcijo $f:[0, 1] \to \R$ definiramo $n$-ti Bernsteinov polinom 
$f_n = \sum_{i= 0} ^n f\left(\frac{i}{n}\right) \cdot B_{n, i} (x)$.
Ker je $f$ zvezna, naj bo $M = \max_{x \in [a, b]} f(x)$.
Potem je
\begin{align*}
  f_n(x) - f(x) &= \sum_{i = 0} ^n f\left(\frac{i}{n}\right) B_{n, i} (x) - f(x) \sum_{i = 0}^n B_{n, i} (x)\\ 
  &= \sum_{i = 0} ^n \left(f\left(\frac{i}{n}\right) - f(x)\right) B_{n, i}(x).
\end{align*}
Od tod sledi $|f_n(x) - f(x)| \leq \sum_{i = 0} ^n \left|f\left(\frac{i}{n}\right) - f(x)\right| B_{n, i}(x).$
Vzamemo $\varepsilon > 0$ in ker je $f$ enakomerno zvezna, izberemo $\delta > 0$,
da iz $|x - x'| < \delta$ sledi $|f(x) - f(x')| <\frac{\varepsilon}{2}$.
\begin{align*}
  \sum_{i = 0} ^n \left|f\left(\frac{i}{n}\right) - f(x)\right| B_{n, i}(x)
  &= \sum_{i,\ \left| x - \frac{i}{n}\right| < \delta} \left|f\left(\frac{i}{n}\right) - f(x)\right| B_{n, i}(x)\\
  &+\sum_{i,\ \left| x - \frac{i}{n}\right| \geq \delta} \left|f\left(\frac{i}{n}\right) - f(x)\right| B_{n, i}(x).
\end{align*}
Sedaj ocenimo drugi člen:
\begin{align*}
  \sum_{i,\ \left| x - \frac{i}{n}\right| \geq \delta} \left|f\left(\frac{i}{n}\right) - f(x)\right| B_{n, i}(x) &\leq \sum_{i,\ \left| x - \frac{i}{n}\right| \geq \delta} 2M B_{n, i}(x).\\
  &\leq 2M \sum_{i = 0} ^n \frac{\left(x - \frac{i}{n}\right)^2}{\delta^2} B_{n, i} (x)\\
  &= \frac{2M}{\delta^2} \frac{x(1-x)}{n} \leq \frac{2M}{n \delta^2}.
\end{align*}
Izberemo še tak $n$, da je $\frac{2M}{n \delta^2} \leq \frac{\varepsilon}{2}$.
Dokaz za $[a, b]$ pa dobimo s preprostim raztegom $[0, 1] \to [a, b]$.

\begin{opomba}
  Uporabili smo identiteto $\sum_0 ^n \left(x - \frac{i}{n}\right)^2 B_{n, i} (x) = \frac{x(1 - x)}{n}$.
\end{opomba}

Weierstrassov izrek je posplošil Marshall Stone na aproksimacijo funkcij v $C(X)$, kjer je $X$ poljuben prostor.
Prostor $C(X)$ je realna algebra, elemente lahko seštevamo, množimo in tvorimo $\R$-linearne kombinacije.
V $C(\R)$ so polinomi $\R[x]$ najmanjša unitalna podalgebra, ki vsebuje $f(x) = x$.
Weiertrassov izrek pravi, da je $\R[x]$ gosta v $C([0, 1])$.

\begin{izrek}[Weierstrass]
  $\R[x]$ je gosta podalgebra v $C(\R)$ glede na enakomerno konvergenco na kompaktih.
\end{izrek}

Izberemo nekaj funkcij $f_1, f_2, \dots \in C(X)$. Ni težko videti, da najmanjšo unitalno podalgebro,
ki vse te funkcije vsebuje, tvorijo vsi izrazi oblike $p(f_{i_1}, \dots, f_{i_n})$, kjer je $p \in \R[x_1, \dots, x_n]$.

\begin{izrek}[Stone-Weierstrass]
  Če je $A \subseteq C(X)$ unitalna podalgebra, ki loči točke $X$, potem je $\overline{A} = C(X)$ v kompaktno-odprti topologiji.
\end{izrek}

\begin{opomba}
  Algebra $\mathcal{A}$ loči točke $X$, če za poljubna $x \neq x'$ v $X$ obstaja $f \in \mathcal{A}$, da je $f(x) \neq f(x')$.
  Dovolj je, če $\mathcal{A}$ vsebuje eno injektivno funkcijo.
\end{opomba}

Pri dokazu si pomagamo z naslednjo oceno.

\begin{lema}
  Funkcija $\varphi(t) = \sqrt{t}$ je na $[0, 1]$ enakomerna limita polinomov.
\end{lema}

\begin{dokaz}
  Razvijemo $\sqrt{t} = \sqrt{1 + (t-1)}$ v Taylorjevo vrsto na $(0, 2)$ in pokažimo, da konvergira za $t = 0$ 
  in da je konvergenca enakomerna na $[0, 1]$.
\end{dokaz}
  Če je $f \in A$ in $f \geq 0$, potem je $\sqrt{f} \in \overline{A}$.
  Ker je $|f| = \sqrt{f^2}$, je tudi $|f| \in \overline{A}$.
  Takoj sledi, da za $f, g \in A$ velja tudi $\min\{f, g\} \in \overline{A}$ in $\max \{f, g\} \in \overline{A}$.
  Če $A$ loči točke, potem za poljubna $x \neq y$ v $X$ in za $a, b \in \R$
  obstaja $f \in A,$ da je $f(x) = a$ in $f(y) = b$.
\begin{dokaz}[Dokaz izreka]
  Naj bo $f \in C(X)$ poljubna.
  Baza kompaktno-odprte topologije na $C(X)$ so množice $\langle f, K, \varepsilon \rangle$,
  kjer je $f \in C(X)$, $K \subseteq X$ kompaktna in $\varepsilon > 0$.
  Fiksiramo $K$ in za vsak par $u \neq v \in K$ izberimo $h_{u, v} \in A$,
  da je $h_{u, v} (u) = f(u)$ in $h_{u, v} (v) = f(v)$.
  Fiksiramo še $u \in K$ in opazujemo množice 
  $$U_v = \{x \in K\ |\ h_{u, v} (x) < f(x) + \varepsilon\}.$$
  Ker je to odprta množica v $K$ in velja $v \in U_v \subseteq K$,
  je $\{U_v\ |\ v \in K\}$ odprto pokritje kompakta $K$.
  Torej imamo končno podpokritje $U_{v_1}, \dots, U_{v_n}$. Definiramo 
  $h_u (x) = \min \{h_{u, v_1} (x), \dots, h_{u, v_n} (x) \}$.
  Tako velja $h_u \in \overline{A}$, $h_u (x) < f(x) + \varepsilon$ za vse $x \in K$ 
  in $h_u (u) = f(u)$.
  Sedaj podobno definiramo še 
  $$V_u = \{x \in K\ |\ h_u (x) > f(x)-\varepsilon\}$$
  in to je odprto pokritje za $K$, torej imamo podpokritje $V_{u_1}, \dots, V_{u_n}$.
  Tako definiramo $h(x) = \max \{h_{u_1} (x), \dots, h_{u_n} (x)\}$ in $h \in \overline{A} \cap \langle f, K, \varepsilon \rangle$,
  kar smo hoteli dokazati.
\end{dokaz}

\begin{posledica}
  \begin{itemize}
    \item Polinomi so gosti v $C(X)$ za poljuben $X \subseteq \R$.
    \item Polinomi več spremenljivk so gosti v $C(X)$ za poljuben $X \subseteq \R^n$.
  \end{itemize}
\end{posledica}

\begin{posledica}
  Funkcije oblike $f(x) = a_0 + \sum_{k = 1} ^n a_k \cos (kx) + b_k \sin (kx)$ so goste v $C((-\pi, \pi))$.
  To je direktna posledica Stone-Weierstrassovega izreka, saj $\{\sin x, \cos x\}$ ločita točke na 
  $[-\pi, \pi]$, zato so linearne kombinacije $\cos^{k} x$ in $\sin^k x$ goste v $C([-\pi, \pi])$.
  Funkciji $\cos^k x$ in $\sin^k x$ pa sta tudi sami linearni kombinaciji funkcij $\sin(kx)$ in $\cos (kx)$.
\end{posledica}








\end{document}